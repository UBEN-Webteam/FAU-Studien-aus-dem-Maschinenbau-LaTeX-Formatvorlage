% !TeX program = lualatex
% !TeX spellcheck = de_DE
% !BIB TS-program = biber
%%%%%%%%%%%%%%%%%%%%%%%%%%%%%%%%%%%%%%%%%%%%%%%%%%%%%%%%%%%%%%%%%%%%%%%%
% This template is a specific instance of the general template for the
% FAU University press available at: https://
%%%%%%%%%%%%%%%%%%%%%%%%%%%%%%%%%%%%%%%%%%%%%%%%%%%%%%%%%%%%%%%%%%%%%%%%
% The above meta-comments are evaluated by some TeX editors like 
% TeXstudio and TeXmaker. They assure, that your editor always calls 
% the correct tools and uses correct settings.
%%%%%%%%%%%%%%%%%%%%%%%%%%%%%%%%%%%%%%%%%%%%%%%%%%%%%%%%%%%%%%%%%%%%%%%%
% For debugging you can get a list of all used packages at the end
% of the log file by calling \listfiles
%\listfiles
%%%%%%%%%%%%%%%%%%%%%%%%%%%%%%%%%%%%%%%%%%%%%%%%%%%%%%%%%%%%%%%%%%%%%%%%
% Loading the document class.
% For the bibliography and the list(s) of acronyms you can use standard
% package biblatex and acronyms resp. or use your own files. 
% The template uses a .bib file with a biblatex style and longtable
% environments for different lists of acronyms, symbols and formulas.
% If you would like to add a separate file for the bibliography or use
% package acronym, please consult the documentation of the general FAU
% template for setting the correct options.
%%%%%%%%%%%%%%%%%%%%%%%%%%%%%%%%%%%%%%%%%%%%%%%%%%%%%%%%%%%%%%%%%%%%%%%% 
\documentclass[%
   paper        = 17x24,%
   language     = ngerman, % english
   bibliography = true, % none, true, false, split
   acronym      = false, % true,combined,split,onlyabbreviation,onlysymbol,noabbreviation,nosymbol
]{faupress-mb}

% You can add global exceptions/restrictions for TeX's hyphenation algorithm depending on the language
% using "-" to mark potential hypehnation points, e.g. Ur-instinkt 
\begin{hyphenrules}{ngerman}
\hyphenation{%
}
\end{hyphenrules}
\begin{hyphenrules}{english}
\hyphenation{%
}
\end{hyphenrules}
%%%%%%%%%%%%%%%%%%%%%%%%%%%%%%%%%%%%%%%%%%%%%%%%%%%%%%%%%%%%%%%%%%%%%%%%
\begin{document}
% Metadaten für Titelseiten einlesen/Load metadata for titlepages
% !TeX root = Hauptdatei.tex
\firstname{Vorname/First name}
\lastname{Zuname/Last name}
\degree{[bitte akad. Grad des/r Verfassers/in angeben]}
% OR for degrees to be printed in front of the name: 
% \degree*{Dipl.-Ing.}
\origin{Geburtsort/Place of birth}
\title{Titel der Arbeit/Title of the work}
%\subtitle{Untertitel der Arbeit/Sub title of the work}
\institute{Dissertation aus dem Lehrstuhl für xxxxxxxxx}
\supervisor{Prof. Dr.-Ing. xxxxxx xxxxxx}
\series{FAU Studien aus dem Maschinenbau}
%\volume{}
\doi{10.25593/978-3-96147-XXX-X}
\isbn{978-3-96147-XXX-X}
\eisbn{978-3-96147-XXX-X}
\issn{xxxx-xxxx}
\printinformation{%
   Druck: xxxxxx
}
\oralexam{mm.dd.yy}
\dean{Prof. Dr.-Ing. habil. Andreas Paul Fröba [für Arbeiten, die nach „alter“ PromO durchgeführt werden – bitte ggf. bei Promotionsbüro fragen – oder:] Prof. Dr.-Ing. Knut Graichen  [für Arbeiten, die nach „neuer“ PromO ab 12/2020 durchgeführt werden]}
\reviewer{%
   Prof. Dr.-Ing. xxxxxx xxxxx\\
   Prof. Dr.-Ing. xxxxx xxxx}
\yearofpublication{2020} %TODO: Fehlt in Doku zur Klasse!

% Haupttitelseite ausgeben/Output titlepage
\maketitle

\frontmatter
% Spezielle Fakultätstitelseite ausgeben/Output special faculty title page
\makefacultytitle

% Vorwort/Preface
% !TeX root = Hauptdatei.tex
% !TeX spellcheck = de_DE
%
\begin{preface}
xxxxxxxxxxxxxxxxx xxx xxxxxxxxxxx
xxx xxxxxxxxxxx xxxxxxxxxxxxxxxxx xxxxxxxx xxxxxxx xxxxxx xxxxxxxxx xxx xxxxxxxxx xx xxxxxxxxxxxxxx xxxxxxxx xxx xxxxxxxxxxxxxxxxx xxx xxxxxxxxxxxxxxxxxxxx xxxxxx xxx xxxxxxxxxxxxxxxxxxxxxxxxxxxxxxxxxxxxxxx xxxxxxxxxxxxxxxxxx xxx xxx xxxxxx xxxxxxxxxxxxxxxxxxxxxxxxxx xxxxxxxxxxxxxxx xxx xxxxxxxxxxxxxxxxxxxx xxxxxxxxxxxx xxxxxxxxxxxxxxxxxxxxxxxxxxxx xxx xxxxxxxxxxxx xxxxxxxxxxxxxx xx xxxxxxxx

xx xxxxxx xxxxxx xxxxxx xxx xxx xxxxxx xxxxxxxxxxxxxx xxx xxx xx xxxxxxxxxxx xxx xxx xxxxx xxx xxxxxxxx xxxxxx xxxxxx xxxxxxxxxxx xxxxxx xxx xxx xxxx xxxxxxxx
xxxxxxxxxxxxxxxxxxx

xxxxx xxxxxxxx xxxxxx xxxxxxxxxxxxxxxxxxxxx xxxxx xxxxxxxx xxxxxx xxxxxxxxxxxxxxxxxx xxxxx xxxxxxxx xxxxxx xxx xxxxxxxxxxxxxxxxxxxx xxxxx xxx xxxx xxxxxxxxxxxxxxxxxxxxx xxxxxxxx xxxxxxx xxxxxxx	xxxxxxxxxxxx xxxxx xxxxxxxx xxxxxxx xxxxxxxxxxxxxxxxxxx xxxxxxxx xxxxxxxx xxxxxxxxxxxxxxxxxxxxx xxxxxxx xxxxxxxxxxxxxxxxxxxx xxxxx xxxxxxxxxxxxxxxxxxx xxxxxxxxx xxxxxxxxxxxxx	xxxxxxxxxxxx xxxxxxxx xxxxx xxxxxxxxxxxxxxxxxxxxxxxxxx xxxxxxxx xxxxxxx xxxxxxxxxxxxxxxxxxx xxxxxxxx xxxxxx xxxxxxxxxxxxxxxxxxx xxxxxxxx xxxxxxxxxxxxxxxxxxx xxxxxx xxxxxxxxxxxxxxxxxxxx  xxxxx

xxxxxxxx xxxxxxx xxxxxxxxxxxxxx xxxxxxxxx xxxxxx xxxxxxxxxx	xxxxxxxxxxxx xxxxxxxx xxxxx xxxxxx xxxxxxxxxxxxx xxxxx xxxx xxxxxxxxxxxxxxx xxxxxxx xxxxxxxxxx xxxxxxxxxxxx xxxxxxxx xxxxxx xxxxx xxxxx xxxxxxxxxxxxx xxxxxxx xxxxxxxxxx xxxxxxx xxxxxxxx xxx xxxx xxxxxxxxxxxxxx xxxxxxx xxxxxxxx xxxxxxxxx xxxxxxxx xxxxxx xxxxxxxxxxxxxxxxx xxxxxx xxxxxx xxxxxxxx xxxxxxx xxxxxx xxxxxxx	xxxxxxxxxxxx xxxxxxx xxxxxxxxxxxxxxxxxx xxx xxxx xxxxx xxxx xxxxxxx	xxxxxxxxxxxx xxxxxxxx xxxxx xxxxxx xxxxxxxxxxxxxxxxx xxx xxxxxxxx xxxxxx xxxxxxxxxxxxxxxxxxxxxx xxxxxxxx xxxxxxx xxxxxxxx

xxxxxx xxxx xxxxxx xxxxxxxxx xxxx xxxxxxxx xxxxxxxxxxxx xxxxx xxxxxxxxxx xxxxxx xxxxxxxxxxxx xxxxxx xxxxxxxx xxx xxxxxxxxxxxxxxxx xxxxxxxxxxxxxxxxxxx xxxx xxxxxxx xxxxxxxxxxxxx xxx xxxxx xxxxxx xxxxxxxxxxx xxxxxxx xxxxxxxxxxxxxxxx xxxxxx xxxxxxxxxxxxx xxxxxx xxxxxx

xxxxxxxxx xxxxxxx xxxxxxxxxxx xxxxxxxxxx xxxxxxxxxx xxxx xxxxxxxxxxx xxxxx xxxxxxxx xxxxxxxxxxx xxxxxxx xxxxxxxxxxxx xxxxxxx xxxxxx xxxxxxxxxxxxxxxxxxx xxxxx xxxxx xxxxxxxxxxxxx xxxxxxx xxxxxxxx xxxxxxxxxxxx xxxxxx xxxxxx xxxxxxx xxxxxxx xxxxx xxxxxxxxxxxxxxxxxx xxxxx xxxxx

xxxxxxx xxxxxxxxxxxxxxxxxxxx xxxxxx xxxxxxxxxxxxxxxxxxxxxxxxxxx xxxxxxx xxxxxxxxxxxxxxxxx xxxxxx xxxxxxxxxxxxxxxxxxxxxxx xxxx xxxxxxxxxxxxxxxxx xxxxxxx xxxxxxxxxxxxxxxxxxx xxxx xxxxxxxxxxxxxxxxxxxx xxxxxx xxxxxxx

xxxxxx xx xxxxxxx xxxxxxxxxxxx xxxxxx xx xxxxxxxxxx xxxxxxxxxx xxxxxx xx xxxx xxxxxxxxxxxxx xxxx xx xxxx xxxxxxxx xxxxxx x xx xxxxxxxxxxxxxxxx xxxxxx xx xxxxxxxxxxxxxxxxxx xxxxxxxxxxxxxxxx xxxxxxx xx xxxxx xxxxxx xxxxxxxxx xxxxxxx xxxxxxxxxxx xxxxxxx xxxxxxxxxx xxxxxxxxxxx xxxxxx xxxxxxxxxxxx

xxxxxx xx xxx xx xxxx xx xxxxxx xxxxxxxxxxx xxxx xx xxxxxxx xxxxxxxxxxx xxxxxxxxx xxx xxxxxxxxxxxx xxxxxxxxx xxxxxxxxxxx xxxxxxxxxxxx xxxxxx xxxxxxxxxxxxxx xxxxxxx xxxxxxxxxxxxxxxxxx xxxxxxxxxxxxxxxxx xxxxxxxx xx xxxxxxxxx xxxxxxxxx xxxxxx xx xxxxxxxxxxxx xxxxxx xxxxxxxxxxxx xxx xxxxxxxxx xxxxxxxx xxxxxxx xxx xxxxxxxx xxxxxxxx xxxxxxxxx xxxxx xxxxxxxxxx xxxxxxxxx xxxxxxxxxxxxxxx xxxxxx xxx xxxxxxxxxxxxxxx xxxxxxxxxxxxxxxxx xxxxxxx xx xxxxxxxxx xxxxxxxxxxx xxx xxxxxxxxxxxxxx xxxxxxxxx xxxxxxxxxx xxxxxxx xxxxxxxxxxx xxxxxxxxx xxxxx xxxxxxxxxxxxxxxx xx xx xxxxxxxxxxx xxxxxxxxx xxxxxxx xxxxxxxxxxxxxx xxxxxx

xxx xxxx xxxxxxxxxxxx xxx xxxxxxxxxxx xxxxxxxxxxxxx xxxx xxxxxx xxxxxxx xxxxxxxxxx xxx xxxxxx xxxxxxxxx xxxxxx xxx xxxxxx xxxxx
\end{preface}

% Inhaltsverzeichnis/Table of Contents
\tableofcontents

% Datei mit Abkürzungslisten einfügen. Alternativ \faupressprintacronyms für Paket acronym/
% insert ready-to-use lists of acronyms. Call \faupressprintacronyms instead for acronym package
% !TeX root = Hauptdatei.tex
% !TeX spellcheck = de_DE
%
\addchap{\acroname}\label{faupressacronyms}

\begin{longtable}{@{}L{0.15\textwidth}@{\hspace{\columnsep}}L{0.15\textwidth}@{\hspace{\columnsep}}L{0.7\textwidth-2\columnsep}@{}}
 \textbf{\textit{Symbol}} & \textbf{\textit{\unitname}} & \textbf{\textit{\descriptionname}} \\
 \endhead
 \endfoot
%
xxxx xxxx xxx & xxx & xxxxxxxx xxxxxxxxxxxxx xxxxxxxxxx xxxxxxxxxxxxxx \\
xxxxxx xxxxxx xxxxx & xxx & xxxxxxxxxxxx xxxxxxxxxxxxx xxxxxxxxxx xxxxxxxxxxxxxx \\
xxxxxxxxx & xxx & xxxxxxxxxxxxxxxxx xxx xxxxxxxxxxxxxxxxxxxxxxxxxx xxx \\
 & & xxx xxxxxxxxxxxxxxxx xxxxxxxxxxxxxxxx \\
xxxxxxx & xxx & xxxxxxxxxxxxxxxxxxxxxxx xxx xxxxxxxxxxxxxxxxxx \\
 & & xxxxxx xxxxxxxxxxxxxx xxxxxxxxxx \\
xxxxxxxx & xxx & xxxxxxxxxxxxxxxxxxxxxxx xxx xxxxxxxxxxxxxxxxxxxxxx \\
 & & xxxxxx xxxxxxxxxxxxxx xxxxxxxxxxxx \\
xxx & xxxxxxx & xxxxxxxxxxxxxxxxx xxxxxxxxxxx \\
xxx & xxx & xxxxx \\
xxxxxxxxx & xxx & xxxxxxxxxxx xxx xxxxxxxxxxxx \\
xxxxxxxxx & xxx & xxxxxxxxxxxxx xxx xxxxxxxxxxxx \\
xxx & xxx & xxxxxxxxxxxx \\
xxx & xxxxxxx & xxxxxxxxxx \\
xxxxx & xxx & xxxxxxxxxxxxx \\
xxxxx & xxx & xxxxxxxxxxxxx \\
xxxxx & xxx & xxxxxxxxxxxxxxxxxxx x xxxxxxxxxxxxxx \\
xxxxx & xxx & xxxxxxxxxxx \\
 & & xxxxxxxxxxxxxxxxxxxxxxxx x xxxxxxxxxxxxx xxxxxxx xxxxxxxx \\
xxxxxxxxx & xxx & xxxxxxxxx xxxx xxxxxxxxxxxxxxx xx xxxxxxx xxxxxxxxxxxxxxxxxxx xxxxx
xxxxxxxxx xxxxxx xxxxxxxxxxxxxxxxxxxx xxxxxxxxxxxx	xxxxx xxxx xxx
xxxxxxxxxxxxxx xxxxxxxxxxx xxxxxxx xxxxx xxxxxxxxxxx xxxxxxxxx \\
xxxxxx & xxx & xxxxxxxxx xxxx xxxxxxxxxxxxxxx xx xxxxxxx
xxxxxxxxxxxxxx xxxxx xxxxxxx xxxxxxxxxxxxx \\
xxx & xxx & xxxxxxxx \\
xxxxx & xxx & xxxxxxxxxxxx \\
xxxxxx xx & xxx & xxxxxxxxxxxxxx x xxxxxxxxxxxx \\
xxxx & xxx & xxxxxxxxxxxxx \\
xxx & xxx & xxxxxxxxxxx \\
xxx & xxx & xxxxxxxxxxxxxxx x xxxxxxxx xxxxxxxxx xxx xxxxxxxxxxxxxxx xxxxxxxxx xxxx xxxxxxx xxxxxxxxxx xxxxxxxxxxx \\
xxx & xxx & xxxxxxxxxxxx xxx xxxxxxxx \\
 & & xxxxxxxxxxxxxxxxxxxxxxxx x xxxxxxxxxxxxx xxxxxxx xxxxxxxx xxxxxxxxxxxxxxxxxxxxx \\
xxx & xxx & xxxxxxxxxxxxxxxxxx \\
xxxxx & xxx & xxxxxxxxxxxxxxxxxx \\
xxxxxxxxx & xxx & xxxxxxxxxxx xxxxxxxxxxxxx xxxxxxxxxxxxxxx \\
xxxxxxxxx & xxx & xxxxxxxxxxx xxxxxxxxxxxx xxxxxxxxxxxxxxx \\
xxxxx & xxx & xxxxxxxxxxxxxxxxxx \\
xxxxx & xxx & xxxxxxxxx xxxxxxxxxxxxxxxxx xxxxxxxxxxxxxxx \\
xxx & xxx & xxxxxxxxxxxxxxxxx \\
xxx & xxxxxxx & xxxxxxxxxxxxxxxxxxxxxxxxxx \\
xxxx xxx & xxx & xxxxxxxxxxxxxxx \\
xxxx xxxx xxx & xxx & xxxxxxxxxxxxxxxxxxxxx \\
xxxxxx xxxxxx xxxxx & xxx & xxxxxxxxxxxxxxxxxxxxx xxx xxxxxxxxxxxxxxxxxxxxxxxxxx \\
xxx & xxx & xxxxxxxxxxxxxxxxxxxxxxx xxxxxxxxxxxxxxx \\
xxx & xxx & xxxxxxxxxxxxxxxxxx \\
xxxx xxxx xxx & xxx & xxxxxxxxxxxxxx xxx xxxxxxxx xxx xxxxxxxxxxxxxxxxxxxx \\
xxxxx & xxx & xxxxx xxxxxx xxx xxxxxxx xxxxxxxxxxxxxxxxxxxxxx \\
xxxxx & xxx & xxxxxx xxxxxx xxx xxxxxxx xxxxxxxxxxxxxxxxxxxxxx \\
xxx & xxx & xxxxxxxxxxxxxxxxxxxxx xxxxxxxxxxxxxxxxxx \\
xxx & xxx & xxxxxx xxx xxxxxxxxxxxxxxx xxxxxxx \\
xxx & xxx & xxxxx xxxx xxxxxxxxxxxx \\
xxx & xxxxxxxx & xxxxxxxxxxxxxxxx \\
xxx & xxx & xxxxxxxxxxxxxxxxxxxxx xxxx xxxxxxx xxxxxxxxxx xxxxxxxxxxx \\
xxxxx & xxx & xxxxxxxxxxxxxxxxxxxx xxxxxxxxxxxxxxx xxxxx xxxxxxxxxxxxxxxxxxxx xxxxxxxxxxxxxxxxxxxxxxxxxxxxxx \\
 & & xxxxxxxxxxxxx xxxxxxx xxxxxxxxxxxxxxxxxxxxxxxxxxx \\
xxxxxxxxx & xxx & xxxxxxxxxxxxxxxxxxxx xxxxxxxxxxxxxxx xxxxx xxxxxxxxxxxxxxxxxxxx xxxxxxxxxxxxxxxxxxxxxxxxxxxxxx \\
 & & xxxxxxxxxxxxx xxxxxxx xxxxxxxxxxxxxxxxxxxxxxxxxxx \\
xxx & xxx & xxxxxxxxxxxxxxxxx x xxxxxx xxx xxxxxxxx x \\
 & & xxxxxx xxx xxxxxxxxxxxxxxx \\
xxx & xxx & xxxxxxxxxxxxxxxxxxxxxxxxxxxx \\
 & & xxxxxxxxxxxxx xxxxxxxxxx xxxxxxxxxxxxxx \\
xxx & xxx & xxxxxx xxxxxxxx xxxxxxxxxxxxxx \\
 & & xxxxxxxxxxxxx xxxxxxxxxx xxxxxxxxxxxxxx \\
xxx & xxx & xxxxxxxxx xxxxx xxx xxxxxxxxxxxxxxxxx \\
xxxxx & xxx & xxxxxxxxxxxxxxxxxxxx xxxxxxxxxxxxxxxxxxxxxx \\
xxxx & xxx & xxxxxxxx xxxxxxxxxxxxxxxx \\
 & & xxxxxxxxxxxxx xxxxxxxxxx xxxxxxxxxxxxxx \\
xxx & xxx & xxxxxxxxxxxxxxxxxx xxxxxxxxxxxx \\
xxx & xxx & xxxxxxxxx xxxxxxxxxxxxxxxxxx \\
xxx & xxx & xxxxxxx xxxxxx xxxxxxxxxxxxxxxxx \\
xxx & xxx & xxxxxxx xxxxxx xxxxxxxxxxxxxxxxx \\
xxxx xxxx xxx & xxx & xxxxxxxxxxx xxxxxxx \\
xxxxxxxxxxxxxxxxxxxxxxxxxxx & xxx & xxxxxxxxxxxxxx xxxxxxxxxx xxx xxx xxxx xxx \\
xxxx xxx & xxx & xxxxxxx																									 x\\
\end{longtable}


%\faupressprintacronyms

% Abbildungs- bzw. Tabellenverzeichnis/
% List of Figures and List of Tables
% both lists are optional -> comment if not wanted
\listoffigures
\listoftables

\mainmatter
% Alle Kapitel werde in einzelnen Dateien abgelegt und hier eingelesen/
% We put each chapter in a separate file and include them in the main file
%%%%%%%%%%%%%%%%%%%%%%%%%%%%%%%%%%%%%%%%%%%%%%%%%%%%%%%%%%%%%%%%%%%%%%%%
% !TeX root = Hauptdatei.tex
% !TeX spellcheck = de_DE
%
\begin{introduction}
%
xx xx xxxxxxxx xxxxxxxxxxxxxx xxx xxx xxxxxxxxxxx xxxxx xxxxxxxxxxxxxxxxxxx xx xxxxxxxx xxxx xxxx xxxxxxxxx xxx xxxxxxxxxxxxx xxx xxxxxxxxxxxxxxxxxx xxx xxxxxxxx xxxxxxxxxxxxxxxxxxx xxxx xx xxxxxx xxxxxxxxxxxxxxxxx xx xxxx xx xxx xxxxxxxxxxxxxxxxxxx xxxx xxxxx xxxxxxx xxxxxxxxxxxxxx xxxxxxxxxxxxxxxxx xxx xxxxxxxxxxxx xxxxx xxx xxxxx xxxxxxx xxxxxxxxxxxxxx xxxxxxxx xxxxxxxxx xxxxxx xxxxxx xxxx xxx xxx xxxxxxxxxx xxx xxxxxxxxxxxxxxxxxxx xxx xxxxxxxxxx xxxxxxxxxx xx xxxxxxxxx xxx xxx xxxxxxxxxxx xxx xxxxx xxxxxxxxxx xxxxx xxxxx xxxxxxxxxx xx xxxxxxxxx xx xxxxxx xxxxxxxx xxxxxxxxxxx xxxx xxx xxxxxxxxxxxxxx xxx xxxxx xxxx xxx xxxx xxxxxxxxx xxxxxxxxx xxxxxx xxxxxxx xxxxxxxxxxxxx xxxxxxx xxx xxxxxxxxxxx xxxxxxxxx xxxxxxxxx xxxxxxxxx xxxxxxxx xxxxxxxxxxx xxxxxxxxxxxxxxxxx xx xxx xxxxxxxxx xxxxxxxxxx xxxxxxxxxxxxxxx xxxxxxxxx xxxxxx xxxxxxxxxx xxxxxxxxxx xxxxxxxxx xxxx xxxxxxxxxxxxx xxxxxxxx xxxxxxxxxxx xxxxxxx xxxxxxxxx xxxxxxxxxxxxxxxxxxxxx xxxxxxxx xxxxxxxxxxxxxxxx xxxxxx xxxxxxxxx xxxxxxxxx xxxxxxx xxxxx xxxxxxxxxxxxx xxxxxxxxxxxxxxxxx xxxxxxxxxxx xxxxxxxxxxxxxx xxxxx xxxxxx xxxxxxx xxxxxxxx xxxxxxxxx xxxxxxxx xxxxxxxxxxx xxxxxxxxxxx xxxxxxxxxxxx xxxxxxxxxxxxxxx xxxxxxxx xxxxxxxxx xxxxxxxxxx xxxxxxxxxxx xxxxxxxxxxxxxxxx xxxxxxx xxxxxxxxxxxxxxxxxxxx xxxxxxxxxxxxxxxxxxx xxxxxxx xx xxxxxxxxxxxxxxxxxx xxxxxxxx xxxxxxxxxxxxxxxxxxx xxxxx xxxxxxxxxxxxxxxxxxxx xxxxxxx xxxxxxxxxxxxxxxxxx xxxxxx xxxx xxxxxxx xxxxxxxx xxxxxxx xxxxxxxxxxxxxxxxxx xxxxxxxxxxxx xxxxx xxxxxxx xxxxxxxxxxxxxxxxxx xxxxxxxxxxxxxxxxxxxxxxx xxxxxxxxxxxxx xxxxxxxxxx xxxxxxxxxxxxxx xxxxxxxxx xxxxxxxxxxxxxxxxxxxxxxx xxxxxxxx xxxxxxxx xxxx xxxxxxxxx xx xxxxxxxxx xxxxxxxxxxxxxx xxxxxxxxx xxxxxxxxxxxxxxxx xxxxxxxxxxxxxxxxx xxxxxxxx xxxxx xxxxxxxxxxxxxxxxxxxx  xx xxxxxxxxxxxxx xxxxxxx xxxxxxxxxxxxxxxxxx xxxx xxxxxxxxxx xxxxxxxxxx xxxxxxxxx xxxxxxxxxx xxxxxxxxx xxx xxxxxxxxx xxxxxxxxxxxxxx xxxx xxx xxxxx xxx xxxxx xxxxxxxxxxxxxxxxx xx xxxxxx xxxxxxxxx xxxxxxxxxxxxx xxxxxxxxxxx xxxxxxxxxx xxxxxxxxx xxxxxxxxxxxxxxxxxxx xxxxxxxxxxxxx xxx xxx xxx xxxxxxxxxxx xxxxxxxxxxxxxxx xx xxx xxx xxxx xxxxxxxxxxx xxx xxxxx xxxxxxx xxxxx xxxxxxxxx xxxxxxxxxxx xxxxx xxxx xxxxxxxxxxxxxxxxx xxxxxxxx xxxxxxxxx xxxxxxxxxxxxxxxx xxxxxxx xxxxxx xxxxxxxxxxxx xxxxxxxx xxxxxxxx xxxxxxxxx xxxxxx xxxxxxxxxx xxxxxxxx xxxx xxxxxxxxxxxxxxx xxxxxxxx xxxxxxxxxxxxx
\begin{figure}
	\centering
	%\includegraphics[width=1.0\textwidth]{./pics/xxxxx}
   \caption{xxxxxxxxx xxxxxxxxx xxxxxxxxxxxxxxx xxxxxxxxxxxxxxxxxxx xx xxxxxxxxx xxx xxx xxxxxxxxx xxx xxx xxxxxxxxxxxxxxxxxxx xxxx xxxxxxxxxxxx xxx xxxxxxxxxxxxx}
   \label{xxxx}
\end{figure}

xx xxx xxxxxxxxxx xxxxx xxxx xxx xxxxx xxxxxxxxx xxxxxxx xxxxx xxx xxxxxx xxxxxxxxxxxxxxxx xxxxxxxx xxx xxxxx xxxxxxxxxxxxx xxxxxxxxxxxxxx xxxx xxxxxxxxxx xxx xxxxxxxxxxx xxxx xxxxxxxxxx xxxxxx xxxxxxxxxxxxxxxx xxxxxxxxx xxxxxxxxxxxxxx xxxxxxxxxxxxxx xxxx

xxx xxx xxxxxx xxxxx xxxxxxxxxxxxx xxxxxxxxxxxxx xxxxx xxxxxxxxxxx xxxxxxxxxxxxx xxxxxxxx xxx xxxx xxxxxxxxxxxxxxxx xxxxxxxxxx xxxxx xxxxx xxxxx xxxx xx xxx xxxxx xxx xxxxxxxxxx xxx xx xxxxx xxxxxxxxxxx xxxxxxxxxxxxxxxxx xxxxx xxxxx xxxxxx xxxx xxxxxxx xxxxxxxxxx xxxxxxx xxxxxx xxxxxxx xx xxxx xx xxxxxxxxxx xxxxxxxxxxx xxx xxxxxxxxxxxxxxxxxx xxxxx xxxxxxxxxxxx xxxxxx xxxx xxx xxxxxxxx xxx xxxxxxxxxxxxxxxxxx xxxx xxxxxxxxxxx xxx xxxx xxxx xxx xxxxxxxx xxxxxxxxxxxxxxxxxxx xxxxxx xx xxxxxxxxxxxxxx xxxxxxxxxxxx xxx xx xxx xxxxx xxx xxxxxxxxxxxxxxxxxx xxxxxxx xxxxxxxxx xx xxxxxxx xxx xxx xxxxxxxxxx xxxxx xxxxxxxxxx xxxxxxxxx xxxxxxxxxxxxxx xxx xxx xxxxxx xxxxxx xxx xxxxxxxxxxxxxxxxx xxxxxxxxxxx xxxxxxxxx xx xxxxxxx xxx xx xxxxxx xxxxxxxxxxx xxxxxxxxx xxxxxxx

xxx xxxxxx xxx xxxxxx xxxxxxxxxxxxxx xxx xxxx xxxxxxxxxxxxx xxxxxxx xxxxx xxxxxxxx xxxxxxxx xxxx xxx xxxxxx xxxxx xxx xxxxxxxxxxxxxxx xx xxxxxx xxx xxxxxxxxxxxxx xxxxxx xxxx xxx xxxxx xxx xxxxxxxxxxx xxxxxxxx xx xxxxxxxxxx xxx xxx xxxxxxxxxxxxx xxxxxxxxx xxxxxxxxxx xxxxxxx xxx xxxxxxxxxxxx xxxxx xxxxxxx xxx xxxxxxxxxxxxxxxx xxxxx xxxxxxxxxxx xxxxx xxxxx xxxx xx xxxxxx xxxxxxxxxxxxxxxxxxx xxxxxxxxxxxxxxxxxx xxx xxx xxxxxxxxxxxxx xxx xxxxx xxx xxx xxxxxxxxxxx xxxxxxxxx xxxxx xxxxxxxxxxxxxxxxxxxxxxxxxxxxxx
\begin{figure}[!ht]
	\centering
	%\includegraphics[width=1.0\textwidth]{xxxx}
	\caption{xxxxxxxxxxxx xxxxxxxxxxxxxxxx xxx xxxxxxxxxx xxxxxxx xxxxxxxxx xxxxxxxxxxxx xxx xxxxxx xxxxxx xxxxxxxxxxxx xxxxx xxxxxx xxxxxxxxxxxx xxxxxxxxxx xxx xxxxxxxxxx xxxxxxxxx}
   \label{fig:xxxxx}
\end{figure}

xxxxxxxx xxxxxx xxxx xxxxxxxxxxx xxxxxxxxx xxx xxxxxxxxxx xxxxxxxxxxxx xxxxxx xxxxx xxxxxxxx xxx xxxxxxxxxxxxxxx xxxxxxxxxx xxxxxxxxx xxxxxxxxxxxxxxxxxxxxxxxxxxxxxxxxxxxxxxxx xxxxxxxxxxxxxxx xxx xxxxxxxxxxxxxx xxxxxxxxx xxxxxx xxxxxxxxxx
\begin{figure}
	\centering
%	\includegraphics[width=1.0\textwidth]{xxxxxx}
	\caption{xxxxxxx xxxxxxxxxxxx xxxxxxxxx xx xxxxxxxxxxxx xxxxxxxxxxxxxxxx}
   \label{fig:xxxxxx}
\end{figure}

xxx xxxxxxxxxxx xxxxxx xxx xxxxxxxxx xxxxxxxxx xxxxx xxxxxxxxxxxxxx xxxxxxxxxxxx xxxxxx xxxx xx xx xxx xxxxx xxxxxx xxxxx xxxxxxxxxxx xxx xxxxxxxxxxx xxxx xx xxx xx xxxxx xxx xxxx xxxxx xxxxxxxxx xxx xxxxxxxxxxx xxx xxxxxx xxxxxxx xxx xxxxxxxxx xxx xxxxxxx xxxxxxxxx xxxxxxx xxxxx xxxx xxxxxxx xxxxxx xx xxxxxxxxxxxxxx xxxxxx xxxxxxxxx xxxx xx xxxxxx xxx xxxxxx xxxxxxxxxxxxxxxx

xxxx xx xxxxxxxxxxxxxxxxxxxxxxxxxxxxxxxxxxxxxxxx xxxxxxxxxxxxx xxxxxxxxxxxx xxxxxxxxxx xxxxx xxxx xxxxxxx xxxxxx xxxxxx xxxxx xxx xxxxxxxxxxx xxxxxxxxxxxxx xxx xxxx xxxxx xxxxxxx xxxxx xxx xxxxxxxxxxxxx xxx xx xxx xxxxxxxx xx xxxx xxx xxxx xxxxxxxxxxxxxxxxxxxxx xxxxxxxxxx xxxxxxxxxxxx xxxxx xxxxxxxxxxxx xxxx xxx xxxxx xxxxxxxxxxxxxx xxxxxxxx xxxxxxxxxx xxxxxxxxx xxx xxxxxx xx xxx xxxxxxxxxxxxxxxxxx xxxxx xxx xxxxxxxxxxx xxxxxxxxxxxxxxxxxx xxxxxxxx xxxxxxxxxxxxxx xxxx xxxxxxxxxxxxxxxxxxxxxxxxxxxxxxx

xxxxxxxxxx xxx xxx xxxxxxxxxxx xxxxxxxxxxx xxx xxxxxxxxxx xxx xxxxxxxxxxxxx xxxxxxxxxxxxxxxxxxxxxxx xxxxxxxxxx xxxxx xxxxxxxxxxxx xxxxxxxxx xxxxxxx xxx xxx xxxx xxxxxxxxx xxxx xxx xxxxxxxxxx xxxxxxxxx xxxxxxx xxx xx xx xxxxxx xxx xxxxxxxxx xxxxxxxx xxxx xxxxxxxxx xxxxxxxxxxx xxx xxxxxxxxxxxx xxxxxxxxx xxxxxxxxxxxx xxx xxxxxxxxxxxxxxx xxxxxxxxx xxxxxxxxxx xxxxxxxxxxxxxxxxxxxxx xxxxxxxxxxxxxxxxxx xxx xxxxx xxxxxxxxxxxxxxx xxxxxxxxxxxxxxxxxxx xxxxxxxxxxxxxxxxx xxxxxxxxxxxxx xxxxx xxxxxxxxxxxxxx xxxxxxxxxxxxx xxxxxx xxx xxxxxx xxx xxxxxxxxxxxxxxxxxxxxxxxxx xxxxxxxxxxx xxxxxx xxxxxxxxxx xxxxxxxxxxx xxxx xxxxxxxxxxxxxx xxx xxxxxxxxxxx xxx xxxxxxxxxxxxx xx xxxxx xxx xxxxxxx xxxxxxxxx xxx xxxxxxxxxxxxxxx xxxxxxxxxxxxxxxxx xxxx xxxxxxxxxxxxxxxxxxxxxxxxxxxxxxx xxx xxx xxxxxxxxxxxxxx xxxxxx xxxxxxxxxxxxxxxxxxxxxxxxx xxxxxx xxxxx xxx xxxxxxxx xxxx xxxxxxxxxxxxxxxx xxxxxxxxxxxxxxxxxx xxx xxxxx xxx xxxxxxx xxxxxxxxx xxxxxxx xxx xxxxxxxxxx xxxxxxxxxxxxxx xxx xxxxxxxxx xxxxxxxxxxxxxxxx xxxxxxxxxx xxxxxxxx xxx xxxxx xxxxxxxxxxxxxxxxxx xxxxxxxxx xx xxxxxxxxxxx xxxxxxxxxxxx xxx xxxxxxxxxxxxxxxxxxxx xx xxxxxxx xxx xxxxxxxxxxxxxxxxxx xxxxx xx xxxxxxxxxxxxxxxxxxxxxxxxxxxxxxxxxxxxxxxxxxxxxx xxxxxxxxx xxxx xxxxxxxxxx xxxxxxx xxxx xxxxx xxxxxxxxxxxx xxxxxxx xxxxxxxxxx xxxxxxxxx xxx xxxxxxxxxxxx xxx xxxx xxx xxxxxxxxxxxxxx xxxxxxxxxx xxxxxxxx xxxx xxxxxxxxxxxxxxxxxxxxxxxxxxxxxxxx

xxx xxx xxxxxxxxxxxxxxxxxxxxxx xxxxx xxxxxxxxxxx xxxxxxx xxx xx xxxxxxxx xxxxxxxxxx xxx xxxxxxxxxxx xxxxxxxx xxxxxxxxxxxx xxxxxxxxxxxx xxxxx xxxxxxxxx xxxxxxxxxxxx xxxxx xxxxxxxxxxxxxxxxx xxxxxxxxxxxxxxxxxx xx xxxxxxxxx xxx xxxxxxxxx xxxxx xxxxxxx xxxxxxxxx xxx xxxxx xxxxxxxxxxxxx xxx xxxxxx xxxxxx xxxx xxx xxxxxxxxxxxxx xxx xxxx xxxx xxx xxxxxxxxxxxx xxxxxxxxxxxxx xxxxx xxxxxx xxxx xxxxxxxxxxxxxxxx xxxxxxx xxxxxxxxxx xxx xxxx xxx xx xxxx xxxxxxxxxxx xxx xxx xxx xxxxx xxxxxxxx xxxxxxxxxxxxxxxxxx xxxxxxxxxxxx xxxxxxxxx xxxx xxxxxxxx xxxxxxxxxxx xxxxxxxxxx xxxxxxxxxxxxx xxxx xxxxxxx xxx xxxxxxxxxxxxxxxxxx xx xxxxxxxxx xxxxx xxxxxx xx xxxxxxxxx xxx xxxxxxxxxxxxxxxxxxxx xxxxxxxxxxx xxxxx xxxxxxxxxxxxxxxxxxxxxxxxxxxxxxxxxxxxxxx xxxxxxxxxxxx xxx xx xxx xxxxx xxxx xxxxxxx xxxxxxx xx xxxxxxxxxxx xx xxxxxxx xxxx xxx xxxxxxxxxxxxxxxxxxx xxx xxxxxxxxxxxxxxxxxx xxxxx xxxx xxxxxxx xxx xxxx xxxxx xxxx xxxxxxxxxx xxxxxx xxxxx

xxx xxxxxxxxx xxxxxx xxx xxxxxxx xxxxxxxxxxxxx xxxxxx xxxx xxx xxxxxxxxxxx xxxxx xxx xxxxxxxxxx xxxxxxxxxxx xxxxxxxxxxxxxxxxxxx xxxxxxxxxxxxxxxxxxx xxxxxxxxxxx xxxxx xxxx xx xxx xxxxxxxx xxxxxxxx xxx xxxxxxxxxxxxxxxxxx xxxxxxxxxx xx xxxxxxx xxx xxxxx xxxx xxxxxxxxxxxx xxxxx xxx xxxxxxxxxx xxxxxxxxxxxxxxx xxxxxxxxxxxxxxxxxxxxxxxxxxx xxxxxxxxxxxxxx xxxxxxxxxx xxxxxxxxx xxx xxxxxxxxxx xxxxxxxxxxxxxxxx xxx xxxxxxxxxxxxx xxxxxx xxxxxxxx xxx xxxxxxxxxxxxxxx xxxxx xxxxxxxxxxx xxxxxxxxxxxxxxxxxxxx xxxxxxxxxx xxxxxx xxx xxxxx xxxxxxxxxxxxxx xxxxxxxxxxxxxxx xxx xxxxxxxxxxxxxxxxxx xxxxxxxxxxxx xxxxxxxxxx xxxxx xxxxxxxxxxxxx xx xxxxx xxxx xxxxxx xxxxxxxxxxxxxxx xxxxxxxxxxxx xxxx xxxxxxxxxxxxxxxxxxx xxxxxxxx xxxxxxxxxxxxxxx xxxxxxxxxx xxx xxxxx xxxxxxxxxxxxxx xxxxxxxxxxxxxxx xxxxxxxx xxxx xxx xxxxxxxxxx xxxxx xxxxx xxxxxxxxxx xxxx xxx xxxxxxxx xxxxx xxxxxxxxxxxxx xxxxxxxxxxxxxxxxxx xxxxx xxxxxxxxxx xxxxxxxxx xxxxxxxxxx xx xxxx xxx xxxxxxxxxxxxxx xxxx xxxxxxxxxxxx xxxxxxxxx xxxxxx xxx xxxxx xxx xxxxxxxxxxxx xxxxxxxxxxxxxxxxxxx xxxxxxxxxxxxxx xxxxxxxx xxxxxxxxxx xxxxxxxxxx xxxx xxxxxxxx xx xxxxxxxxxxxxxxxx xxx xxxxxxxxx xxx xxxxxxxxxxx xxxx xxx xxxxxx xxxxxxxxxxxxx xxxxxxxxxxxxxxx xx xxx xxxxxxxxxxxxxxxxx xxxxxx xxx xxxxxxxxxx xxxxxxx xxx xxxxxxxxxx xxxxxx xxxxxxxxxxxxxx xxxxx xxxxxxxxxxxxx xxxxxxxx xxx xxx xxxxxxxxxxxxxxxxx xxxxxxxxxxxxxx xxxxxxxx xxxxxx xxxxxxxxxxx xxxxxxxxxx xxxxx xxxx xxx xxxxxxxxxxxxxxxxxxx xxx xxxxxxxxxx xxxxx xxx xxxxxxx xxx xxxxxxxxxxxxxxxxx xxxxxxx xxxxxxx

xx xxxxx xxxx xxxxxxxxxxx xxxx xx xxxx xxxxxxxxxxx xxxxxxxxxxx xx xxx xxxxxxxx xxxxxxxxxxxxxxxxxx xxxx xxxxxxxx xxx xxxxxxxx xxxxxxxx xxxxx xxx xxxxxxx xxxxxxxxxxxxxxxxxxxx xxxxxxxx xxxxxxxxxx xxx xx xxxxxxxxxxx xxxx xxxxxxxxxxxxxxxxxxxxxxxxxx xxxxxxx xxxxx xxx xxxxxxx xxx xxxxxxxxxxxxxxxxxxxxxx xxxx xxxxx xxxxxxxxxxxxx xxxxxxxxxxxx xxx xxxxxxxxx xxxxxxx xxx xxxxxx xxxxxx xxxxx xxxx xx xxx xxxxxx xxx xxxxxxxxxx xxxx xxxxxxxxxxxxxxxxxxxxxxxxxxxxxxx xxxxxxxxxxx xxxxxx xxxxxxxxxxxx xxxxxx xxxxxxxxxxxxxxxxx xxx xxxxxxxxx xxxxxxxxxxx xxx xxxxxxxxxxxxxxxxxxxxx xxxxxxxxxx xxxx xxxxxxx xxxx xxx xxxx xxxxx xxxxxx xx xxxxxxxxxxxxxxxxxxxxx xxx xxxxxxxxxxxxx xxxxxxxxx xxx xxxxxxxxxx xxxxxxxxxxxxxxxxxxx xxxxxxxxxxxxxxxxx xxxxxxxxxxxxxxxxxxxx xxx xxxx xxxxxxxxxxxxxxxxxx xxxxxxxxxxxxxxx xxxxxxxx xxxx xxxxxxxxxxxxxxxxxxxxxxxxxxxxxxx xxxxxxxxxxxxxxxxxxxxxxxxxx xxx xxxxxx xxxx xxxxxx xxxx xxxxx xxx xxxxxxxxxxxxxxxx xxxxxxxx xxx xxxxxxxxxxxxxx xxx xxxxxxxxxxxxx xxx xxx xxxxxxx xxxxxxxxx xxxxxxx xxxx xxxxxxxxxxxxx xxxxxx xxxxxxx xxxxxxxxxxxxxxxx xxxxx xxxxxxx xxx xxxxxxxxxxxxx xxxxxxxxxxx xxxxxxxxxxxxxx xxxxxx xx xxxxxxxxxxxxxxxxx xxxxxxx xxxxx xxxx xxxxxxxxxxx xxxxxxxxxxx xxx xxxxxxxxxxx xxx xxxx xxxxxxx xxxxx xxx xxxxx xxx xxxxxxx xxxxxxx xxx xxxxxxxx xxx xxx xxx xxx xxxxxxxxxxxxxxxxxxxxxxxx xxxxxxxxxxx xxx xxxxxxxxxxxx xxxx xxxxxxxxxxxxxxxxxxxxxxxxxxx xxxxxxxxxxx

xxx xxxxxxxxxxx xxxxxx xxxx xxxx xxxxxxxxxx xxx xxxxxxxxxxxxxxxxxxxxxxx xxxxxxxxxxx xxxxxxxxxxxxxxxxxxxxx xx xxxxxxxxxxxxxxxxx xxxxxxx xxxx xxx xxxxxxxxxxxx xxxxxxxxx xxxxx xxxxx xxx xxxxxxxxxxx xxx xxxxxxxxxxx xxxxx xxxxxxxxxxxxxxxxxx xxx xxxx xxxxx xxxxxxxxxxxxxxx xx xxxxxxxxxxxxx xxxxxxxxxxxxxxxxxxxxxx xxxxxx xx xxxxxx xxx xxxxxxxxxx xxxxxxxx xxx xxxxxxxxxxx xxxxxxxxxxxxxxxxx xxxxx xxxxxxxx xxxxxxxxxxxx xxx xxxxx xxx xxxxxxxxx xxxxxxxxxxxx xxxxxxxxxxx xxx xxxxxx xxxxxx xxxx xxx xxxxxxxxxxxxxxxx xxx xxxxx xxx xxx xxxxxxxxxx xxxxxxxxxx xxxxxxxxx xxxx xx xxxxxxxxxxxxx xxxxxxxxxxxxxxxx xxx xx xxxxxx xxxxxx xxxx xxx xxx xxxxxxxxx xxxxx xxx xxxxxxxxx xxx xxxxxxx xxx xxxxxxxxxx xxxxxxxxxxx xxxxxxxxxxxx xxx xxx xxxxxx xxxxxx xxxxxxxx xxxxxx xxxx xxx xx xxxxxxxxxxxxx xxxxxxxxxxxxxxxxxxxxxxxxxxx xxxxxxxxxxxx xxxxxxxxxxxxxxx xxxxxxxxx xxxxx xxxxxxxxxxxxxxxxxxxxxxx xxxxxxxxxxxx xxxxxxxxxxxxxxxxxxxxx xxx xxxxxxxxxxxxx xxxxxxxxxxxxxxxxxxxxxxx xxxxxxx xxxx xxxxxx xxxxxxxxxxxx xxx xxx xxxxxxxxxxx xxxxx xxxxxxxx xxx xxxxxxxxxxxxxxxxxxxxxxxx xxxxxxxxxxxx xxxxxxxxxxxxxxxxxxxxxx xxxxxx xxxxxxxxx xxxxxx xxxx xx xxxxx xxx xxxxxxxx xxxxxxxxxxxx xxx xxx xxxxxxxxxxxx xxxxxxxxxxxx xxxxxxxxxx xxxxxx xxx xxxxxxxxx xxx xxxxxxxxxxxxxxx xxxxxxxxxxx xxx xxxxxxxxxxxxxxxxxxx xxx xxxx xxxxx xxxxxxxxxx xxx xxx xxxxxxxxx xxxxxxxxxxxx xxxxxx xx xxxxxxxxxxxxx xxxxxxxxxxxxxxxxxxxxxxxxxxxx xxxxxxxxxxxx xxx xxxxx xxx xxx xxxxxxxxxx xxx xxxxxxxxxxxxxxx xxxxxxxxxxxxxx xxxxxxx xxxxx xxxxxxxxxxxxxxxxx xx xxxxxxxxxxxxx xxxxxxxxxxxxxxxxxxxxxxx xxx xxxxx xxx xxxxxxxxxxxx xxx xxxxxx xxxxxxxx xxxxx xxx xxx xxxxxxxxxxx xxxxxxx xxx xxxxxxxxx xxxxx xxxxxxxxxxxxxxxxxx xxx xxx xxxxxxxxxxx xxx xxxxxxxxxxxxxxxxxxxxxxxx xxxxxxxxxxxx xxxxxxxxxxxxxxxxxxxxx xx xxx xxxxxxxxxxxxxxxxxxxxxxxxxx xx xxxxxxxxxxxxx xxxxxxxxxxxxxxxxxxxxxxxxxxxxx xxxxxxxxxxxxxxxx xxxx xx xxxxxxx xxx xxxx xx xxxxxx xxxxxx xxxxxxxxxxxxxx xxxxxxxxxxxxx xxxxxxxxxxxxxxxxxxxxx xxxxxx xxxx xxxxxxxxxxxxxxx xxx xxxxxxxxxxxx xxxxxx xxxxxxxx xxx xxxx xxx xxxxxxxxxx xxxxxxxxxx xxx xxxxxxxxxxxx xxxxxxxxxxxxxxxxxxxxx xxxxxxxxxxxx
%~~~~~~~~~~~~~~~~~~~~~~~~~~~~~~~~~~~~~~~~~~~~~~~~~~~~~~~~~~~~~~~~~~~~~~~~~~~~~~~~~~~~~~~~~~~~~~~~~~~~~~~
\section{Abschnittsüberschrift}\label{sec:xxxxx}
%~~~~~~~~~~~~~~~~~~~~~~~~~~~~~~~~~~~~~~~~~~~~~~~~~~~~~~~~~~~~~~~~~~~~~~~~~~~~~~~~~~~~~~~~~~~~~~~~~~~~~~~
xxxxxx xxxxx xxxxxxxxxx xxxxxxxxxxxxx xxxxxxxxxxxx xxxx xxxxxxxxxxx xxxxxxxxxxxx xxxxxxxxxxx xxxxxxxxxx xxxx xxxxxxxx xxxxxxxxx xxx xxxxxxxxx xxxxxxxxxxxxxxxxxx xxxxx xxxxxxxxxx xxx xxxxxxxxxxx xxxxxxxxxxx xxx xxxxxxxxxxxx xxxxxxxxxxx xxxxxxxxxxxxxx xxxxxxxxxxxxxxxxxxxx xxx xxxxxxxxxxxx xxxxxxxxx xxxx xx xxx xxxxxxx xxx xxxxxxxxxxx xxxxxxxxxxxxxxxxxxxxxxx

xxxx xxxxx xx xxx xxxxxxxxx xxxxx xxxx xxxxxxxxx xxxx xxx xxxxxxx xxx xxxxxxxxxx xx xxxxxxx xxx xxxxxxxxxxx xxxxxxxxxxx xxxxxxxxxxxx xxxxxxxxx xxxxxxxxxx xxxxxxxxx xxxxxxxx xxxxxxxxxx xxxx xx xxx xxxxxxxxxxxxx xxxxxxxx xxxxxx xxx xxxxxxxxxxxxxxxxx xxxxxxxxxxxxxx xxxxxxxx xxxxxxxxxxxx xxxxxxxxxxxx xxxxxxx xxx xxxxxxxxxxx xxx xxxxxxxxxxxxxx xxxxx xxxxx xxxxxxxx xxxxxxxxxxxx xxxxxxxxxxxxxxxxxx xxxxxxx xxxx xxxxxxxxxxxxxx xxxxxx xxxxxx xxxxx xxx xxxxxx xx xxx xxxxxxx xxx xxxxxxxxxxxxxxxxxxxxxxxx xxxxxxxxxxxx xxxxxxxxxxxx xxxx xxx xxx xx xxxxxx xxxxxx xxxxxxxxxx xxxxxxxxxxxx xxxxxxxxxxxx

xxxxxxxxxxxxxxxxx xxxxx xxxxxxxxxxxxxxxxx
xx xxxxxxxxxxxxxxxxxxxxxxxxx xxxx xxx xxxxxxxxxxxxxxxxxxxxxxxxx xxxxx xxxxxxxx xxxxxxxxx xxx xxx xxxxxxxxx xxxxxxxxxxxx xxxxxxxxxxxx xxx xxxxxxxxxxxxxxxxx xxxxxxx xxxxxxx xxx xxxxxx xxx xxxxxxxxxxxxxxxxxxxxxxxx xxxxxxxxx xxxxxxxxxxxxx xxxxxxxxx xxxxxxxxx xxxxxxxxxx xxxx xxxxx xxxxxxx xxxxxxxxxxxx xxxxxxxx xxx xxxxxxxxxxxxxxxx xxxxx xxxxx xxxxxxxxxx xxxxxxxx xxx xxxxxxxxxxxxxxx xx xxxxx xxx xxxxxxxxxxxxxxxxxxxxxxxxxxx xxxxxxxxx xxxxxxx xxxx xxx xxxx xxx xxxxxxxxxxxx xxxxxxxxxxxxx xxxxxxxxxxxxxxxxxxxxxxxxx xxx xxxxx xxxxxxxxxxxxxxxxxxxxxxxxxxx xxxxxxxxx xxxxxxxxxxxxx xxxxxxxxxxxxx xxxx xxxxxxxxxxxxxxxxxxxx xx xxxxxxxxx xxxx xxx xxxxx xxxxxxxxxxxxxxxxxxxxxxxxxxxxx xxxxxxxxxxxxxxxxxxxxxxxxxxxxxxx xxxx xxxxxxxxxxxxxxxxxxxxxxxxxxxx xxx xxx xxxx xxx xxxxxxxxxxxxxxx xxx xxx xxx xxxxxxxxxxxxxxxx xxxxxxxxxx xxxxxxxxxxxxx xxxx xx xxxxxx xxxxxxx xxxxx xxxxxxxxxxxx

xxxxxxxxxxxxxxxxxxxxxxxxxxx xxxxx xxxxxxxxxxxxxx xxx xxx xxxxxxxxxxxxx
xxx xx xxxxxxxxxxxxxxxxxxxxxxxx xxxxxxxxxxxx xxxxxxxxx xxxx xxxx xxxxxxxxxxxx xxx xxxxxxxxxx xxxxx xxxxxxxxxxxx xxxx xxx xxxxxxxxxxxxxxxxxx xxxxx xxxxxxxxxx xx xxxxxxxxxxx xxx xxxxxx xxxxxxxx xxxx xxxxxx xxx xxxxxxxxxxxx xxxx xxxxxxxxxxxxxx xxxxxxxx xxx xxxxxxxxxx xxxxxxxxxxxxxxxxx xxxxx x xxxxxxxx xxx xxx xxxxxxxxxxxxxxx xxxxxxxxxxxxxxxxxx xxxxx x xxxxxxxx xxxxxxxxxx xxx xxxx xxxxxx xxxxxxxxxx xxxxxxxxxxxxx xxxxxxx xxxxxxx xxxxx xxx xx xxx xxxxxxxxx xxxx xxx xxxxxxxx xxxxxxx xx xxxxx xxx xxxxxxxxxxxxx xxxxxxxxxxx xx xxxxxx xxxxxx xxx xxxx xxx xxxxxxx xxxxxxxxxxxxxx xxx xxx xxxxxx xxxxxxxxxxxxxxx xxxxxxxxxx xxxx xxxxxxxxxxxxxxxxxxxxxxxxx

xxxxxxxxxxxxxxxxxxxxxx xxxxxxxxxxxx xxx xxxxxxxxxxxxxxxxxxxxx
xx xxxxxxxxxxxxxxxxxxxxxxxxxx xxx xxxxxxxxxxxxxxxxxxxxxxxxxxxxxx xxxx xxx xxx xxxxxxxxxxxxx xxxxxxxxxxxx xxx xxxxxxxxxxxxxxxxxxxxxxxxxxx xxxxxxxxxx xxxxxx xxxxxxx xxxxxxxxxxxxxxxxxxx xxxxx xxxxxxx xx xxxxxxxxxx xxxxxxx xxxxxxx xxx xxxxxxx xx xxx xxxxx xxxxxxxx xxxxxxxxxxx xxx xxx xxxxxxx xx xxx xxxxxxxxxxxxx

xxxxxxxxxxxxxxxxxxxxx xxx xxxxxxxxxxxxxxxxxxxx xxxx xxx xxxxxxxxxxx
xxxx xxxxxxxxxxxxxxxxxxxxxxxxx xxxx xxxx xxxxxxxxxxxx xxxx xxx xxxxxx xxxxxxxxx xxxxxxx xxxx xxx xxxx xxx xxxxxxxxxxxxx xxxx xxx xxxxxxxxxxx xxx xxxxxxxxxxx xxxxxxxx xxx xxxxxxxxxxxxxx xxxxx xxxxxxxxxx xxxxxxxx
x
xxxxxxxxxxx
x
xxxxxxxxxxxxxxxxxxxxx xxxxx xxxxx xxxxxxxx xxxxxxxxxxxx xxxxxxxxxxxxxxxxxx
xx xxxx xxxxxxxxx xxxxxxxxxxxx xxxxxxx xxxx xxxx xxxxxx xxx xxxxxxxxxxxxxxx xxxxx xxxxxxxx xxxxxxxxxxxx xxxxxxx xxxxxxxxxxxxxx xxxxxxxx xxxx xxxxxxxx xx xxxxxxxx xxx xxxxxxxxxx xxxxxxxxxxxxx xxxxxxxxx xxxxxxx xxxxx xxxxxx xxxxxxxxxxxxxx xxxx xxxx xxxxxxxxxxxxx xxxxxxxx xxxxxxxx xxxxxxxxxxxxxxxxxxxx xxxxxxxxxxxxx xxxxxxxxxx xxxxxxxxx xxxxxxxxxxxxxxxxxx xxxxxxxxxx xxxxxxxxxxx xxxxxxxxxxxx xxxxxxxxxxxxx xxxxxxxxxxxxxxxxx xxxxxxx xxxxxx xxxxxxxxxxxxxx xxxx xxxxxxxxxxxx xxxxxx xxx xxxxxxxx xxx xxxxx xxxxxxxxx xxxxxxxxxxxx xx xxx xxxxxxxxx xxxx xx xxxxxxxxxxxxxxx xxxxxxxxx xxxxxxx xxxxxxxxxxxxx xxxxxxx xxxxxxxxxx xxxx xxxxxxxxxxx xxx xxx xxxxxx xxx xxxxxxxxxxxxxxxxxxxx xxxxxxxxx xxxxx xxxxxxxxxxxxx xxx xxxxx xxxxxxxxxxx xxxxxxxxxxx xx xxxxxxxxxxxxxx xxxxx xxxxxx xxxxxxxx xxxxxxxxxx xxxxx xxxxxxxx xxx xxx xxx xxxxxxxxxxxx xxxxxxxxxxxxxx xxx xxxxxxxx xxxxxxxxxxx xxx xxx xxxxx xxx xxxxxxxxxx xxxxxx xxxxxxxxx xxxxxx xxxxxxx xxxx xxxxxxxxxxxxxxxxxxxxxxxxxxx xxxxx xxxxxxxx xxxxx xxxxxxx xxxxxxx xxxx xxxx xxxxxxxxxx xxxx xxxxxxx xxxxxxxxx xxx xxxxxxxxxxx xxxxx xxxxx xxxxxxxxxx xxxxxx xx xxxxxxxxxxxxx xxxxxxx xxxxx xxxxxxxxxxxx xxxx xxxxxxxxxx xxxxxx xxxxxxxxxx xxxxxxxx xxxxxxxx xxxxxxxxxxxx xxxxx xxxxxxxx xxxxxxxxxx xxxxxxx xx xxxxxx xxxxxx xxxx xxx xxxxxx xxx xxx xxxxxxxxxxxxxxxxxxxxxxxxxxxxxxxxxxxx xxxxxxx xxxxxxxxxxxxxxxxxxxx xxxxxxxxx xxxxxxxxxx xxx xxxxxxx xxxxxxxxxxxxxxxx xxxxxxxxxxxxxxx xx xxxxxxxxx xxx xxx xxxxxxxx xxxxxx xxxxxxxxx xxx xxx xxxxxx xx xxxxxxxx xxx xx xxxxxxxxx xxxx xxxx xxx xxxxxxxxx xx xx xxxxxxxxxx xxxx xx xxxxxxxx xxx xxxxxxxxxxxxxxxxx xxxxxxxxxxx xxxxxxx xxxxxx xxxx xxx xxxxxxxx xxx xxxxxxx xxxxxxx xxx xxxxx xxxx xxxxxxxxxxxxxxxxxxxxxxxxxxx xx xxx xxxxxx xxxxxxx xxx xxxxxxxxxxxxxxx xxxxxxx xxxxxxxxxxxxxxxxxxxx xxxx xxxxxxxxxxxxx xxxxxxxxxxxxxx xxxxxx xxxx xxxxxx xxxxxxxxxxx xxxxxxxxxxx xxxxxxxxxxx xxxxxxxxx xxxxx xxx xxxxxxxxxxxx xxxxxxxxx xxxxxxxxxxxxxxx xxxxxxxxxxx xxxxxx xxxx xxx xxx xxxxxxxxx xxxxxxxxxxxxxx xxxxxxxx xxx xxxxxxxxx xxxxxxxx xxxxx xxxxxxxxxxxxxxxxxxxxxxxxxxxxxxxxxxxxxxxxx xxx xxxxxx xx xxxxx xxxxxxxx xxxxxxxxxxxxxxxxx xxxxxxxxxx

xx xxxxxx xxx xxxxxx xxxxxxx xxx xxxxxxxxxxxxxxxxxxxx xxx xxxxxxxxx xxx xxxxxxxxxxxxxxxxxxxxxxxxxxxx xxxx xxxxxxxxxxxx xxxxxxxxx xxxx xxxxx xxx xxxxxxxxx xxxx xxxxxxxxxx xxxxxxx xxxxxxxxxxx xxxxxxxxxxxx xxx xxxxxxxxxxxxx xxxxxxxxxxxxxxxx xxxxxxxxxx xxxxxxxxx xx xxxxxxxxxxxx xxxxxxxxxxxxxxxx xxxxxxxxx xx xxxxxxx xx xxxxxx xxxxxxxxxxxxxxxxxxxxx xx xxx xxxxxxxxxxxxxxx xxxxxxxxxxxxxxxxx xxxxxxxxxx xxxx xxxxxxxxxxxxx xxxxxxxxxxxxxxxxxxx xxxxxxxxxxxxxxxx xxxxxxxxxxxxxxx xxxxxxx xxxxxxxxxxxxx xxxxxx xxxx xxx xxxxxxxxxxxxxxxxxxx xxxxxx xxxxxxxxxxxx xxxxxx xxxxxx xxxx xx xxxxxxxxxxxxxxxxxxxxxxxxxxxxxxxx
%~~~~~~~~~~~~~~~~~~~~~~~~~~~~~~~~~~~~~~~~~~~~~~~~~~~~~~~~~~~~~~~~~~~~~~~~~~~~~~~~~~~~~~~~~~~~~~~~~~~~~~~
\section[head={Evtl. Kurztitel für Kolumnentitel}]{Abschnittsüberschrift}\label{sec:xxxx}
%~~~~~~~~~~~~~~~~~~~~~~~~~~~~~~~~~~~~~~~~~~~~~~~~~~~~~~~~~~~~~~~~~~~~~~~~~~~~~~~~~~~~~~~~~~~~~~~~~~~~~~~
xxx xxxxx xxxxx xxxxxxxxxxxxxxxx xxxxxx xxxxxx xxxxxxxx xxxx xxxxxxxxxxx xxxxx xxx xxxxxxxxxxx xxxxxxxxxxxxxxxxxx xxxxxxxxxx xxxxxx xxxxxxxxxxxxxxxxxxxxxxxxxxxxxxxxxxxxxxx xxxxx xxxxxxx xxxxxxxxxxxx xxx xxxxxxxxxxxxxxx xxxxxxx xxxxxxx xxxxxx xxxxxxxxx xxxx xxxxxxxxxx xxxxxxxxxxxxxxx xx xxx xxxxxxxxxx xxx xxxxxxxxx xxx xxx xxxxx xxxxxxxxxx xx xxxxxxx xx xxx xxxx xxxxxx xxxxxxxxxx xxx xxx xxxxxxx xx xxxxxx xxxx xxx xxxxxxx xxxxxxxx xxxxxxx xxx xxxxxx
\begin{itemize}
	\item xxxxxxxxxxx xxxxxxxxx xxxxxxxxxxx xxx xxxxxxxxxx xxxxxxxxxxxx xxxxxx xxxxxxxxxxxx xxxxxxxxxxxx xxx xxxx xxxxxxxxx xxxxxxxxxxxxxxxxx
	\item xxxxxxx xxxxxxxx xxxxxxx xxxxxxxxxxxxxxxxxx
	\item xxxxxxxxxxxxxx xxx xxxxxxxxxxxx xxxxxxxx xxxxxxxxxx xxxxxxxxxxxxxxx xxxxx xxxx xxxxxxxxxxx xxxxxxxxx xxxxxxxxxxxx
	\item xxxxxx xxxxxxxxxxxxx xxxxxxxxxxxxxxxxxx xxx xxx xxxxxxxxx xxx xxxxxxxxxxxxxx xxxx
\end{itemize}

xxx xxx xxxxxxx xxxxxxxxxxxxxxxxxx xxxxxx xxxx xxx xxx xxxxxx xxx xxx xxxxxxxxxx xxxxxxxxxxxxxxx xxx xxxxxxxxxx xxxxxxxxx xxxxxxxxxxx xxxx xxxxx xxxx xxxxxxxxx xx xxxxx xxxxxxxxxxxxxxxx xxxxxxxxxxxxxx xxxxxx xxxxxxx
\end{introduction}
%%%%%%%%%%%%%%%%%%%%%%%%%%%%%%%%%%%%%%%%%%%%%%%%%%%%%%%%%%%%%%%%%%%%%%%%
% !TeX root = Hauptdatei.tex
% !TeX spellcheck = de_DE
%
\chapter{Stand der Forschung}\label{chap:StandForschungTechnik}
%~~~~~~~~~~~~~~~~~~~~~~~~~~~~~~~~~~~~~~~~~~~~~~~~~~~~~~~~~~~~~~~~~~~~~~~~~~~~~~~~~~~~~~~~~~~~~~~~~~~~~~~
%~~~~~~~~~~~~~~~~~~~~~~~~~~~~~~~~~~~~~~~~~~~~~~~~~~~~~~~~~~~~~~~~~~~~~~~~~~~~~~~~~~~~~~~~~~~~~~~~~~~~~~~
In Abschnitt~\ref{sec:xxxxx} xxxxx xx xxxxxxxxxxxxxxxxxxxxxxxxxxxxxxxxxxxxxx xxx xxxxxxxxxxxxxxxx xxxxxxxxxxxx xxx xxxxxxxxxxxxxxxx xxxxxxxxxxx xxx xxxxxxxxxxxxx xxxxxxx xxx xxx xxxxxxxxxxxxxxx xxx xxxxx xxxxxxxxxxx xxxxxxxxxxxxxxxx xxxxxxxxxxx xxx xxx xxxxxxxxxxxxxxxx xxx xx xxxxxxxxxxxxx xxxx xxx xxx xxxxx xxx xxxxxxxxx xxx xxxxxxx xxx xxxxxxxxxxxxx xx xxxxxxxxxxxxxxxxxx xxxxxxxxxxx xxxxxxxx xx xxxxxxx xxxx xxxxxx xxxxxxxx xxx xxx xxx xxxxxxxxxxxxxx xxxxxxxxxx xx xxxxxxxxxxx xx xxxxx xxx xxxxx xxx xxx xxxx xx xxxxxxxxxxxxxxxxxxxxxxxxxxxxxxxxxxxxxxxxx xxxxxxxxxxxxx \cite{KEYH} xxxxxxxxx xxx xxxxxxxxxxxxxxxxx xxxxxxxxxx xx xxxxxx

Bild~\ref{fig:xxxx} gibt einen Überblick über die Inhalte dieses Kapitels.
\begin{figure}
	\centering
%	\includegraphics[width=1.0\textwidth]{xxxxx}
	\caption[Ggf. Kurztitel für Bildverzeichnis]{xxxxxxxxx xxx xxxxx xxx xxxxxxxxx xxx xxxxxxx xxx xxx xxxxxxxxxxxxxxxxxxxxxxxx xxxxxx xxx xxxxxxxxxxxxxxxx xxxxxxxxxxx xxx xxxxxxxxxxxxx xxxxxxx xxx xxx xxxxxxxxxxxxxxx xxx xxxxx xxxxxxxxxxx xxxxxxxxxxxxxxxx}
  \label{fig:xxxx}
\end{figure}

xx xxxxxx xxxx xxxxx xxxx xxxx xxxxxxxxxxxx xxxxxxxxxxxx xxx xxxxxxx xxxxxxxx xx xxxx xxxxx xxxxxxxxx xx xxxxxxxxxxxxxxxxxxxxxxxxxxxxxxxxxxx xxxx xxx xxx xxxxxxxxxxxxx xxxxxxxx xxx xxxxxxxxxxxxxxxxxxx xxxxxxxxxxxx xxxxxx xx xxxxx xxxxxxxxxxx xxxx xxxxxxxxxxx xxxxxxxxx xxx xxx xxxxxxxxxx xxx xxxxxxx xxxxxxxxxx xxxxxxxxxxxxx xxxxxxxxxxxx xxxxxxx xx xxxxxxxxxx xxxx xx xxxxxx xxxxxxxxx xxx xxx xxxxxx xxxxxxxxxxxxxxxxxxxxxxxxxxxxxx xxx xxxxxxxxxxxxxxx xxxxxxxxxxxxxxxxxxxxxx xxx xxxxxxxxxxxxxxxxx xxxxx xxxxxxxxxxxxxxxxxxxxxxxx xxx xxxxxxxxxxxxxxxxx xxxxxxxxxxxx xxxxxxx xxxxxxxxxxx xxxxxxxxxx xxx xxx xxxxxxxxxx xxx xxxxxxx xxxxxxxxxx xxxxxx xx xxxxxxxxxxxxxxxxxxxxxxxxxxxxxxxxxxxxxxxxxxxxxx xxxxxxxxxxx xxxxxx xxxxxxxxx xxxxx xxxx xxxxxxxxxx xxx xxxxxxxxxxxxxxxxxxxxxxxxxxx xxx xxxxxxxxx xxxxxxxxxxxx xx xxx xxxx xxx xxxxxxxxxxxxxxxx xxxxxx xxx xxxxxxxxxxxxxxxx xxx xxxxxxxxxxxxxxxxxxxxxxxxxxxxxx xxx xxxx xxx xxxxxxxxxxxxxxxxxxxxx xxxxx xxx xxxxxxxxxxx xxx xxxxxxxxxxxxx xxxxxxxxx xxxxxxx xxxx xxxxx xxxx xxxxxxxxxxxxxx xxxx xxx xxxxxx xxxx xx xxxxxxxx xxxxx xxxxx xxx xxxxxxxxxx xxx xxxxx xxxxxxxxxxxxxxxxxxxxxxxxxxxxx xxx xxxxxxxxxxx xxxxxxxx xxx xxxxxxxxx xxxxx xxxxxxxxxxx xxxxxxxxxx xx xxxxxxxxxxxxxxxxx xx xxx xxxxxxxxxxxx xxx xxx xxxxxxxxxxxxxxxx xxxxxxx xxxxxxxxxxx xxxxxxxxxxxx xxx xxxxxxxxxxxxxx xxxxxxxxx xxx xxx xxxxxxxx xxxxxxxxxxx xxxxxxxxxxxxxxxxxxx xxxxxxx xxx xxxxxxx xx xxxxxxxxxxxxxxxxx xxx xxxxxx xxxxxxxxxxxxxxxxxxxxxxxxxxxxxxxxxxxxxxxxxxxxxx xxxx xxxxx xxxxxxxxx xxxx xxxxxxxxx xxxxxxxxx xxx xxxxxxxxxxx xxx xxxxxxxxxxxxxx xxxxxxxx xxxxxx xx xxxxxx xxxxxxxxxxxx xxxxxxxxxxxxxxxxxxxx xxx xxx xxxxxxxxxxxxxxx xxxxxxxxxxxx xxxxxx xxx xxxxxxxxxxx xxxxxxxxxxx xxxxxxxxxxxxxxxxxx xxx xxxxxxxxxxxxxx xxxxxx xxxxxxxxxxxxxxxxxxx xxx xxxxxxxx xxxxxxxxxxxx xxxxxxxxx xxxxxxxxxxx xxxxxxx xxxxx xxxxxxx xxx xxxxxxxxxxxxxx xxx xxx xx xxxxxxxxxxxxx xxxxxxxxxxxxx xxxxxxxxxxxx xxxxxxxxxxxx xxx xxxxxxxxxxxxx xxxxxxxxxxxxxxx xxxxxxxxxxxxxx xxxxxx xxxxxxxxxxxxxx xx xxxxxxxxxxxxxx xxxxxx xxxxxxxxxxxx xxxxxxxx xxxxx xxxxx xxxx xxxx xxxxxx xxxxxxxxxxx xxxxxxxx xxxxxxxx xxxxxxxxxxxxxxxxxxxxxx xx xxxxxxx xxx xx xxx xxxxxxxxx xxxxxxx xxxxxxxxx xxxxxxxxxxxxxxxxxxxxxx xxxxx xxxxxxxx xxxx xxxxx xxxxxxxxx xxxxxxxxxxx xxxxx xxxx xxxxxxx xxxxxxxxxxxxxx xxxxxxxxxxx xxxxxxxxxxx xx xxxxxxx xxxxxxxxxxxxxxxxxxxxxxxxxxxxxxxxxx xxxxxxx xxx xxxxx xxxxxxxxxxxx xxxxx xx xxxxxxxx xx xxxxxxxx xxxxxx xxxxxxxx xxxxxx xxxxxxxxx xxx xxxx xxxxxxxxxxxxxx xxxxxxxx xxxxx xxx xxxxxxxx xxx xxxxxxxxxxxxx xxxxxxxxxxxxxxx xxx xxx xxxxxxxxxx xxxxxxxxx xxxxxxx xxxxxxx xxxxxxx xxxxxxxxxxx xxxxxxxxx xxxx xxx xxxxxx xxxxxxxxxxxxx xxxxxxxxx xxx xxxxxxxxxxxxxx xxxxxxxxxxxx xxx xxxxxxx xxx xxxxxxx xxxxxxxx xxxxxxxxxxxxxxx xxxxxx xxxx xxx xxxxxxxxxx xxxxxxxxx xxx xxx xxxxxxxxxxxxxxxx xxxxxxxxxxx xxx xxxxxxxxxxxxx xxxxxxx xxx xxx xxxxxxxxxxxxxxxx xxxxxxx xxxxx xxxxx xxxx xxxxxxxx xx xxxx xxx xx xxxxxx xxxxxx xxxxxx xxxxxxxxxxxx xxxxxxx xxx xxxxxxx xx xxxxxxxxxxxxxxxxxxxxxxxxxxxxxxxxxxxx xxxxxxxxxx xxxxx xxx xxxxxxxxxxxx xxxxxxxxxxx xxx xxxxxxxxxxxxx xxxxxxx xxx xxxxxxxxxxx xxxxxxxx xxxxxxx xxxxx xxx xx xxxxxx xxxxxxxxxxxxx xxxxxxxxxxxx xxxxxxxxxxxxxxxxxxxxxxxxx xxxxx xxx xxx xxxxxxxxxxxxx xxxxxxxxxxxxxx xxx xxxxxxxxxxx xxx xxxxx xxxxxxxxxxx xxxxxxx xxxxxxxxxxx xxx xxx xxxxx xxxxx xxxxx xxxxxxx xxxxxxxxxxxxxxxxx xxx xxxxxxxxxxxxxx xxxxxxxxxxxxxxxxxxxxxxxxxxxxxxxx xxxxxxxxxxxxxxxx xx xxxxxxx xxx xxx xxxxxxxxxxxx xxxxxxx xxx xxxxxxxxxxx xxxxxxxxxxxxxxxxxxxxx xxxx xxxxxxxxxxx xxx xxxxxx xx xxxxx xxxxxxxx xxxxxxxx xxx xxxxxx xxxxxxxxxxxx xx xxxxxx xxxxxxxxx xxxxxxx xxxx xxx xxxxxx xxxxx xxx xxx xxxxxxxxxxx xxxxxxxxxxx xxxxxxxxxxxxxxxxxxx xxxx xxxxxxxxx xxxxxxxxxx xxxxxxx xxxxxx xx xxxxxx xxxxxx xxx xxx xxxx xxxxxxxxxxxxxxxxxx xxxx xxxxxxx xxxxx xxxxx xxxxxxxxx xxx xxxxxxxxxxxx xxx xxxxxxxxxxxxx xxxxxxxxxxxxxxxxxxxxx xxxxxxx xx xxxxxxx xxx xxx xxxxxxx xxx xxxxxxx xxxxxxxx xxxxxx xxxxxxxxxxxxxxxx xxxxxxxxxxx xxx xxxxxx xxx xxxxxxxxxxx xxxxxxxxxxxxxxxxx xxxxxxxxxx xxx xxxxx xxxxx xxxxxxxxxxxxxx xxxxxxxxxxxxxx xxxxxx xxxx xxxxx xxxxx xxxxxxxx xxxxxxxxxxxxxxx xxxxxxxxxxxxxxxx xxx xxxxxxxxxxx xxxxxxxxxxxxxxx xxxxxxx xxxxxx xxxx xxxxxxxx xxxxxxxxx xxxxx xxxxxxxxxxxxxxxxx xxxxxxx xxxxxxxxxxxxx xxxxxxxx xxxxxxxxxxx xxxxxxxx xxxxxx xxxxxxxx xxxxxxxxx xxx xxxxxxxxxxxx xxxxxxx xxxxxxxxx xx xxxxxxxxxx xxx xxxxxxxxxxxxxxxxxxxxxxxxx xxxxxxx xxxx xxxx xxx xxxxxxxxxxxxxx xxx xxxxxxxxxxxxxx xxxxxxxx xxx xxxxxxx xx xxxxxxx xxx xxxxxxx xxxxxxxxxxxxx xxx xxxxxxxxxxxxxxxxxxxxxxxxxxxxxxxxxxxxx xxxxxxxxxxx xxxxx
%~~~~~~~~~~~~~~~~~~~~~~~~~~~~~~~~~~~~~~~~~~~~~~~~~~~~~~~~~~~~~~~~~~~~~~~~~~~~~~~~~~~~~~~~~~~~~~~~~~~~~~~
\section{Abschnittsüberschrift}\label{sec:xxxxxxx}
%~~~~~~~~~~~~~~~~~~~~~~~~~~~~~~~~~~~~~~~~~~~~~~~~~~~~~~~~~~~~~~~~~~~~~~~~~~~~~~~~~~~~~~~~~~~~~~~~~~~~~~~
xxx xxxxxxx xxxxxxxxxxxxxxxxxx xxxxxxx xxxxxxxxxxxxxxxxxxx xxxxxxxxxxxx xxxxx xxxxx xxxxxx xxxxxxx xxxxxxx xxxx xxxx xxxxx xxxx xxxx xxxxxxx xxxxxxxxxxx xxxxxxxxxxx xx xxxxxxxxx xx xxx xxxxxxxxxx xxxxxxxxx xxxxxxx xxxxx xxxxxxx xxxxxx xx xxx xxxxxxxxx xxxxxxxxxx xxxx xxxxx xxxxxxxx xxxxxxxxxxx xxxx xxxxxxxxxxxxx xx xxx xxxxxxx xxxxxxxxxxxxxxxxx xxx xxxxxxx xxxxxxxxxxxxxxxxx xxxxx xxxx xxxx xxxxxxxxxxxxx xxxxxxx xxxxxxxxxxxxxxxx xxxxxxxxxx xxxxxxxxxxx xxxxxxxxx xxxx xxxxxxxxxxxx xxxxxxx xxxx xxx xxxxxxxxxxxxxxxxxx xxx xxxx xxx xxxxxxxxxxxxxxxxxxxxx xxx xxxx xxxxxxxxx xxxxxxxxxxxx xxxxxxxxxxxxxxxxxx xxxxx xxxxxxxxx xxxxxxx xxxxxxxxxxxxxxxxxx xxx xxxxxxxxx xxxxxxxx xxxxxxxxxxxxxxxxxx xxxxx xxx xxxxxxxxx xxxxxxxxxxxxxxx xxxxxxxxxxxxxxxxx xxxxxxxxxxxx xx xxxxxxxxxxxxxxxx xxxxxxxx xxx xxxx xxx xxxxxx xxx xx xxxxx xxxxxxxx xxxxxxxxx xxx xxxxxxxxxxx xxxxxxx xxxxxxxxxxxxxx xxxxxxxx xxxxx xxxxx xxxxxxxxxxxx xxx xxxxxxxxxxx xx xxxxxx xxx xxxxxxxxxxxxxx xxxxxxxxxxxxx xxxxxxxxxxxxxx xxxxx xxxxxxxxxxxxxxx xxxx xxxxxxxxxx xxx xxxxx xxx xxxxxxxxxxxx xxxxxx xxxxxxxxx xxx xxxx xxxxxx xxxxxxxx xxxxxxx xxxxxxxxxxxxxxxx

xxxx xxxxxxxxxxxxxx xxxxxxxxxx xxx xxxxxxxxxx xxxxxx xxxx xx xxxxxxxx xxxxxxxx xxxxxxxxxxxxx xxxxx xxxx xxx xxxxxxx xxxxxxxxxxxxxxxxx xxxx xxxxxxxxxxx xxxxxxx xxxxx xxxxxxxxxxxx xxxxxxx xxx xxx xxxxxxxxxxxxxxxxx xxxx xxxxxxxxxxx xxx xxxxxxxxxxxxxxxx xxxxxxx xxx xxxxxxxxxxxxxx xxxxxxxxxx xxxxxxxxxxxxxxxx xxxxxxx xxx xxxxxxxxxxx xxxxxxxxxx xxxxxxxxxxxx xxxxx xxxxxxxx xxxxxxxxxxxx xxxxxxxxxxxxxxxxxxxxx xxx xxxxx xxxxxxxxxxxx xxxxxxxxxxxxxxx xxxxxxxx xxxx xxxxxxxxxxxxxxxxxxxx xxx xxxxxxxxxx xxxxxx xxxxxxxxx xxxxxxxxxxxx xxx xxxxx xxxxx xxxxxx xxxxxxxxxxxxx xxxxxxxxxx xxxxxxxxxxxxxxx xxxxxxxxxxxxxxxxxx
%~~~~~~~~~~~~~~~~~~~~~~~~~~~~~~~~~~~~~~~~~~~~~~~~~~~~~~~~~~~~~~~~~~~~~~~~~~~~~~~~~~~~~~~~~~~~~~~~~~~~~~~
\subsection{Unterabschnittsüberschrift}\label{sec:xxxxxxxx}
%~~~~~~~~~~~~~~~~~~~~~~~~~~~~~~~~~~~~~~~~~~~~~~~~~~~~~~~~~~~~~~~~~~~~~~~~~~~~~~~~~~~~~~~~~~~~~~~~~~~~~~~
xxx xxx xxxxxxxxxxxxxxxxxxxxx xxxxxxx xxx xxxxxxx xxx xxxxxxxxxxxx xxxx xxxxxxx xxx xxxxxxxxxxxx xxxxx xxxx xxxxxxxxxxxx xxx xxxxxxxxxx xxxxxxx xxxxxxxxxxxx xxx xxx xxxxxxxxxxxxxx xxxxxxxxxxxx xxxxxxxxxxxxx xxxxxxxxxxxxxxxxxxxxxxxxxxxxxxxx xxxxx xxxxxxxxxxxx xxxx xxx xxxxx xxxxxxxxxxxxx xxxxxxx xxxxxxxxxxxxxxxxxxxxxxxxxxxxxxxxxx xxx xxxxxxxxxxxxxxxxxxxxxxxxxxxxxxx
\begin{itemize}
   \item xxxxxxxxxxxxx xxx xxxxxxxxxxxx
   \item xxxxxxxxxxxxx xxx xxxxxxxxxxxxxxx
   \item xxxxxxxxxxxxx xxx xxxxxxxxxx xxx xxxxxxxxxxxxxxxxxxxxx
\end{itemize}

xxx xxx xxxxxxxxxxxxxxxxxxxxx xxxxxxxxxx xxxxxx xxx xxxxxxxxxx xxxxxx xxxxx xxxxx xxxxxxxxx xxxxxxxxxxxxxxxxxx xx xxxxx xxxxxxxxxxxxxxxxx xxxxxxxxxx xxxxxx xxxxxxx xxx xxxxxxxxxxxxxxxx xxxxxxxx xxx xxxxxxxx xxxxxxxxxxxxxx xxx xxxxxxx xxx xxxxxxxxxxx xxxxxxxxxxxxxxxxxxx xxxxxxxxxxxxxx xxxxxxxxxx xxx xxxxxxxxxxxxx xx xxxxxxxxxxxxxxxxxxxxx xxx xxxxxxxxxxxxxx xxxx xxxxxxxxxxxxxxxxx xx xxx xxxxxxx xxxxxxxxxxxxxxxxxxx xxx xxxxxxxxxxxxxxxxxxxxxxxxxxxx xxxxxxxxxx xxx xxxxxxxxxxxx xxxxxxxxxxxxx xxxx xx xxx xxxxxxxxxxx xxx xxxxx xxxxxxxxx xxx xxx xxxxx xxx xxxxxxxxxxxx xxxxxxx xxx xxxxxxxxxxxx xxxxxxxxxxx xxx xxx xxxxxxxxxxxx xxxxxxxxxx xxx xxxxxxxxxxx xxx xxx xxx xxxxxxxxxxxxx xxx xxxxxxxxxx xxx xxxxxxxxxx xxx xxx xxxxx xxx xxxxxxxxxxxxxxxxx xxxxxxxxxxxxxxxxxxxxxxxxxxxxxxxxx xxxx xxx xxxxxxxxxxx xxx xxx xxxxxxxx xxx xxxxxxxxxxxxxxxxxxxxxxx xxx xxx xxxxxxx xxxxxxx xxxxxx xxxx xxxx xxx xxxx xxx xxxxx xxxxxx xxxx xxxxxxxxxxxxxxxxxxxxxxx xxxx xxx xxxxxxxx xxxxxxxxxxxx xxxxxx xxxxxxxxx xxxxxxxxxxxx xxx xxxxxxx xxxxxxx xxxx xxxxxxxx xxxxxxxxxxxxxxxxx xxxxxx xxxxxxx xxxxxxxxxxxxxxxxxxxxxxxxxxxxxx xx xxxxxxxxx xx xxxxxxxxxxxxxxxxxxxxxxxxxxxxxxxxxxxxx xxxxx 
\begin{table}
	\centering
   \caption{xxxxxxxxxxxx xxxxxxxxx xx xxx xxxxxxxxxxxxxxxxxxxxxxxxx}
	\begin{tabularx}{\linewidth}{L{3cm} L{0.25\linewidth} L{3cm} X}
		xxxxxxxx xxxxxxx & xxxxxxxxxx xxxxxxx xxxxxxx & xxxxxxxxx xxxxxxx & xxxxxxx xxxxxxxxx \\\hline
		xxxxxxxx xxxxx & xxxxxxxxxx xxxxx xxxxxxxxxx & xxxxxxxx & xxxxx xxxxxxx xxxxxxxx \\\hline
   \end{tabularx}
	\label{tab:xxxx}
\end{table}

\subsubsection*{Unterunterabschnitt ohne Nummerierung}
%~~~~~~~~~~~~~~~~~~~~~~~~~~~~~~~~~~~~~~~~~~~~~~~~~~~~~~~~~~~~~~~~~~~~~~~~~~~~~~~~~~~~~~~~~~~~~~~~~~~~~~~
xxx xxxxxxxxxxxxxxxxxxxxxxxx xxxxxxxxxxxxxxxx xxxx xxxx xxxxx xxx xxxxxxxxxxxxxxxxxxxxxx xxxxxxxxxxxxxxx xxxxxxxxxxx xx xxx xxx xxxx xxxxxxxxxxx xxx xxx xxxx xxxxxxxxxx xxxxxxxxxxxx xxx xxxxxxxx xxxxxxxxxx xxx xxxxxxxxxxxxxxx xxxxx xxxxxxxx xx xxxxxxxx xxxxxxxx xxxxxxxxxx xx xxxxx xxx xxxxxxx xxxxxxxxxxxxx xxxxxxxx xxxxx xxxxxxxxxxxxxxxxxxxx xxxxxx xxxx xxx xxxxxxxxxxx xxxxxx xxxx xxxxxxxxxxxxxxx xxxxxxxxxx xxxxxxx

Beispiel, wie eine Formel anzugeben ist:
\begin{align}
   \varphi=\arctan\left(\frac{A_3-A_1}{A_0-A_2}\right)
	\label{eq:importantFormula}
\end{align}

Im Beispieltext wird anschließend auf die beispielhafte Gleichung~\eqref{eq:importantFormula} verwiesen.
xxxxxx xxxxxxxxxxx xxxxxxxxx xxx xxxxxxxxxxxxxxxxxxxxxxxx xxxxxxxxxxxxxxx xxx xxxxx xxx xxxxxxxxxxxxxxxxx xxx xxx xxxxxx xxx xxxxxxxxxxx xxxxxxxxxx xxx xxxxxx xxx xxxx xxxxxxxxxxxxxxxx xxxxxxxxxx xx xx xx xxx xxxxxx xx xxxx xxxx xxxxxx xxxxx xxxxxxxxxxxx xxxxxxxxxxxxxxxxxxxxx xx xxxxxxxxxxxx xxx xxxxxxxxxx xxx xxxxxxxxxxxx xxx xxxxxxxxx xxxxxxxxxxxxxxx xxxxx
%

\minisec{xxxxxxxxxxxxxxxxxxxxxxx xxxxxxxxxxxxxxxx}
xxx xxxxxxxxxxxxxxxxxxxxxxx xxxxxxxxxxxxxxxx xxxxxxx xxxx xxxxxxxxxx xxx xx xxxxxxxxxxx xxxxxxxxxxxxxxxxx xxx xxxxxxxxx xxxxxxxxxxxxx xxxxxxxxxxxxxxxxxxxxxxxxxxxxxxxx xxxxxx xxx xxxxxxxxxxxxxxxxxxx xxx xxxxxxxxxxxxxxxx xxxxxx xxxx xxxx xxxxxxxxxxxxxxxxx xxxxxxxxxxx xxxxxxxxx xxxxxxxxxxxxxxxxxxx xxx xx xxxxxxxx xxxxxxxx xxxxx xxxxxxxxxxxx xxxxxxxxxxxxxxxx xxxxxxxxx xxx xxxxxxxxxxx xxxxxxxxxxxxx xxxxxxxxxxxxxxxxxxxxxxxxxxxxxxxxxxxx xx xxxxxxx xxxxxx xx xxxxxxxxxxxxx xxxx xxx xxxxxxxxxxxx xxxx xx xxxxxx xxxxx xxxxxxxxxxxxxxxxxxxxxxxxxxxxxx xxx xxxxxxxx xxxxxxxxx xxxxxxxxxxxxxx xxx xxxxxxxxxxxxxxxxx xxxxx xxx xxxxxxxxx xxxxxxxxx xxx xxxxxxxxxxx xxxxxxxxxxxxxxx xxxxx xxxx xxxxxxxxxxxxxx xxxxxxx xxx xxxxxxxxxxxxx xxxxxxxxxxxxxxxxxxxxxxxxxxxxx xxx xxxxx xxxxxx xxxxxxxxx xxxxxxxxxxxxxxxx xxxxxx xx xxxxxxx xxxxxxxx xxx xxxxxxxx xxxxxxx xxxxxxxxxxxxxx xxxx xxxxxxxxxxxxxx xxxxxxxxxxx

%~~~~~~~~~~~~~~~~~~~~~~~~~~~~~~~~~~~~~~~~~~~~~~~~~~~~~~~~~~~~~~~~~~~~~~~~~~~~~~~~~~~~~~~~~~~~~~~~~~~~~~~
\subsection{Unterabschnittsüberschrift}\label{sec:xxx}
%~~~~~~~~~~~~~~~~~~~~~~~~~~~~~~~~~~~~~~~~~~~~~~~~~~~~~~~~~~~~~~~~~~~~~~~~~~~~~~~~~~~~~~~~~~~~~~~~~~~~~~~
xx xxxxxx xxx xxxxxxxxxxxxxxxx xxxxxx xxxx xxxxxxxx xxxxxxxxxxxxxxxxxx xxx xxxxxxxxx xxxxxxxxxxxxxx xxxxxx xxx xxx xxxxxxxxxxx xxxxxxxxx xxxxxxxxxxxxx xxxxxxxxx xxxxxx xxxxxxx xxx xxxxxxxxxxxxxxx xxx xxx \cite{KEYA} xxxxxxxxxxxxxxxxx
%~~~~~~~~~~~~~~~~~~~~~~~~~~~~~~~~~~~~~~~~~~~~~~~~~~~~~~~~~~~~~~~~~~~~~~~~~~~~~~~~~~~~~~~~~~~~~~~~~~~~~~~
\section{Abschnittsüberschrift}\label{sec:xx}
%~~~~~~~~~~~~~~~~~~~~~~~~~~~~~~~~~~~~~~~~~~~~~~~~~~~~~~~~~~~~~~~~~~~~~~~~~~~~~~~~~~~~~~~~~~~~~~~~~~~~~~~
xxxxxxxxxxxx xxxx xxxxxxx xxx xxxxxxxxx xxxxx xxxxx xxxxxxxx xxxxxx xxxx xxxxx xxxxxxxxxx xx xxxx xxxxxxx xxxxxxxxx xxxxxxxxxxx xxx xxx xxxxxxxxxxxxxxxxx xxx xxxxxxxxxxxx xxxxxxxxxxxx xxxxxx xxxxxxxxxxx xxxxx xxxxxxxxxxx xxxxxx xx xxxxx xxxxx xxxxxxxx xxxxxxxxxxx xxxxx xxx xxxx xxx xxxxxxxxxxx xxxx xxxxxxxxxxxxxxx xxxx xxxxxxxxxxxx xxxxxxx xxx xxx xxxx xxxxxxxxxx xxxxxxxxxxxxxxxxxx xxxxxx xxx xxxx xx xxxxxx xxx xxxxxxxxxxx xx xxxxxxxxxx xxxx xxx xxxxx xxx xxxxxxxxxxxxxxxxxxx xxx xxxxxxxxxx xxxxxxx xxxx xxxxxxxxxxxx xxxxxx xxxxxxx \cite{KEYD} xxx xxxxxxx xxxxxxx xxx xxxxxxxxx xxx xxxxxxxxxxxxxxxxxxxxx xxx xxx xxxxxxxxxxxxxxxxxx xxx xxx xxxxxxxxxxxxxxxx xxx xxx xxxxxxxxxxxxxxxxxx xxx xxxxxxxx xxxxx xxxxxxx xxxxxxxxxxx xxxx xxxx xxxxxxxxxxxxxxxxxxxxxxxx xxxxxxxxxxxx xxxxxx xxx xxxxxxxxxxxxxxxx xxx xxxxxxxxxx xxxxxxxxxxxxx xxxxxxxxxxxx xxxxxx xx xxxxxxx xxxxxxxxxxxxxxxx xx xxxxxx xxx xxxxxxxxxxxxxxxxxx xxxxxxxxxxxx xxxxxxxxx xx xxxxxxx xxxx xx xxx xxxx xxxxx xxx xxx xxxxxxxxxxxxxxxxx xxxxxxxxxxxxx xxxxxxxxxxxx xxx xxxxx xxxxxxxxxx xx xxxxxxxxx xxxx xxx xxxxxxxxxx xxxxxxxxxxx xxxxxxxxx xxx xxxxxxxxxxxxxx xxx xxxxxx xxxxxxxxxxx xxxxxxx xxx xxxxx xx xxxx xxxxxxxxxxxxxxx \cite{KEYF} xxxxxxxxxxxxxxx xxxxxxxxxxxxxxxxx xxxxxxxxxxxxxxxxx xxx xxxxxxxxxxxxx xxx xxxx xxxxxxxxxxxx xxxxxxxxxxx xxx xxxxxxxx xxxxxx xxx xx xxxxxxxxxxxxxxxxxxxxxxxx xxxxxxxx

%%%%%%%%%%%%%%%%%%%%%%%%%%%%%%%%%%%%%%%%%%%%%%%%%%%%%%%%%%%%%%%%%%%%%%%%
%\include{}
%%%%%%%%%%%%%%%%%%%%%%%%%%%%%%%%%%%%%%%%%%%%%%%%%%%%%%%%%%%%%%%%%%%%%%%%
%\include{}
%%%%%%%%%%%%%%%%%%%%%%%%%%%%%%%%%%%%%%%%%%%%%%%%%%%%%%%%%%%%%%%%%%%%%%%%
%\include{}
%%%%%%%%%%%%%%%%%%%%%%%%%%%%%%%%%%%%%%%%%%%%%%%%%%%%%%%%%%%%%%%%%%%%%%%%
%\include{}
%%%%%%%%%%%%%%%%%%%%%%%%%%%%%%%%%%%%%%%%%%%%%%%%%%%%%%%%%%%%%%%%%%%%%%%%
%\include{}
%%%%%%%%%%%%%%%%%%%%%%%%%%%%%%%%%%%%%%%%%%%%%%%%%%%%%%%%%%%%%%%%%%%%%%%%
% !TeX root = Hauptdatei.tex
% !TeX spellcheck = de_DE
%
%~~~~~~~~~~~~~~~~~~~~~~~~~~~~~~~~~~~~~~~~~~~~~~~~~~~~~~~~~~~~~~~~~~~~~~~~~~~~~~~~~~~~~~~~~~~~~~~~~~~~~~~
\chapter{Zusammenfassung und Ausblick}\label{chap:xxx}
%~~~~~~~~~~~~~~~~~~~~~~~~~~~~~~~~~~~~~~~~~~~~~~~~~~~~~~~~~~~~~~~~~~~~~~~~~~~~~~~~~~~~~~~~~~~~~~~~~~~~~~~
xxx xxxxxxx xxxxxxxxxxxxxxxxxx xxxxxx xxx xxxxxxxx xxx xxxxxx xxxxxxxxxxx xxxxxxxx xxx xxxxxxxxx xxxxxxx xxxxx xxxx xxxxxx xxxxx xx xxxxxxxxxxxxx xxxxxxxxxxxx xxxxx xxxxxxxxxx xxxxxxxxxx xxxxxxxxxxx xxx xxx xxxxxxxxxxx xxx xxxxxxxxxxx xxx xxxxxxxxxx xxxxxx xxx xxxxxxxxxxxx xxxx xxxxx xxxxx xxx xxxxxxxxxx xxxxxxx xxxxxxxxxx xxx xxxx xxxxx xxxxxxxx xxxxxxxxxxxxx xxxxxxxxx xx xxxxxxxxx xxx xxxxx xxx xxx xxxxxxxxxxxxxxxxxxxx xxx xxxxxxxxx xxxxxxxxx xxxxxxxxx xxxxxxxxx xxxxxxxxxx xxxxxxxx xxxxxxxxxxxx xxxxxxxx xxxxxxxxxxx xxxxxx xxxx xxxxxxxxxxxxx xxxxxxxxxxxxx xx xxx xxxxxxxxxxxxxxxxxx xx xxxxx xxx xxxxxxxxxxxxxx xxxxx xxxxxxxxxxxxxxxxxxx xxxxxxxxxxxxxxxx xxxx xxxxxxxxxxxxxxx xxxx xxxx xxxxxx xxx xxxxxxxx xxx xxxx xxx xxxxxxxxxxx xxxxxxx xxxxxxxxx xxx xxx xxxxx xxx xxxxxxxxxxx xxxxxxxxxxxxxxxxxxx xxxxx xxxxxxxxxxxxxxxxxxxx xxxxx xx xxxxxx xxxxxx xxx xx xxx xxxxxxxxxxxxxxx xxxxxxxx xxxxxxxxxxxx xxxxxx xxx

xxx xxxxxxxxxxx xxx xxxxxxxxxxxxxxxxxxxxx xxxxxxxxx xxxxx xx xxxxxxx xxxxx xxxxxxxxxx xxxxxxxxxxxx xxxxxxxxx xxxxx xxxx xxxxx xx xxxxx xxx xxxxxxxxxxxxxxx xxxxxxxx xxxxxxxxxxxxx xxxxxxxxxxxx xxxxxxxxx xxx xxxxxxxxxxxx xxxxxxxxxxxxxx xxxxxxxxxxx xxxx xxxxx xxxxxxxxxxxx xxxx xx xxxxxxxxxxxx xxx xxxxx xxxxx xxxxxxxxxx xxx xxx xxxxxx xxxx xxxx xxx xxx xxxxxxxxxx xxx xxxxxxxx xxxxxxxxxxxx xxx xxxxxxxxxx xxx xxxxxxxxxxxxxx xxxxxxxxxxxx xxx xxxxxxxxx xxxxxxxxx xxxxxxxxxx xxx xxxxxxxxxxxx xxxxxx xxxxxxxxxxxx xxxxxxxx xxxx xxxxxx xxxxx xxx xxxxxxxxxxx xxxxxxxxxxxxx xxx xxxx xxxxx xxxxxxx xxxxx xxx xxxxxxxxxxxxx xxx xx xxxxxxxx xx xxxx xxx xxxx xxxxxxxxxxxxxxxxxxxxx xxxxxxxxxx xxxxxxxxxxxx xxxxxxx xxx xxxxxxxx xxxxxxxxxxx xxxxxxxx xxxxxxxxx xxxxx xxxxxxxxx xxxxxx xxxxx xxxxxxxx xxxx xxxxxxxx xxxxxxxxxxxx xx xxxxxxxxxxxx xxxxxxxxxxx xxxxxxxxxxxxxxxxxxxxxxxxxx xx xxxxxxx xxxxx xxxxxxxxxxxxxxxx xxx xxxxxxxxxx xxxxxxxxxxx

xxx xxxxxxxxxxx xxxxxxxxx xxx xxxxxxxxxx xxxxxx xxxxxxxxxx xxxxxx xxx xxxx xxxxxx xxxxxxx xx xxx xxx xxxxx xxxx xxxxxxxx xxx xxxxxxxxxx xxxxxxxxxxxxxxxxxxx xxxxxxxxxx xxxxx xxxxx xxxxxxxxxxxxxxxxxxxxxxxx xxxxxx xx xxxxxxxxxxxxx xxxxxxxxxxxx xxx xxxxxxxx xxx xxxxxxxxxxxxx xxxxxxxx xxxxxxxxxxxx xxxxx xxxx xx xxx xxxxx xxxxx xxxxxxx xxxx xxxxxxxx xxx xxxxxxxxxxxx xxxxxxxxx xxxxxxxxxxxxxxxxxxxxx xxx xxxxx xxxxx xxxxxxx xxxxxxxxxx xxxxxxxx xx xxxxxxxxxx

xxx xxx xxxxxxxxx xxxxxxxxxxxx xxxxxxxxxxxxxxxx xxxx xxx xxxxxxxxxx xxxxxxxxxxxx xxxxxxxxxxx xxxxxxxx xxxxxxxxxxxx xxx xxx xxxxxxxxxxx xxxxxx xxx xxxxxx xxxxx xxxxxxxx xxxxxxxxxxxx xxx xxxxxxxxxxxx xxxxxxxxxxxx xxxxxxxxxx xxx xxxx xxx xxxxxxxxxxx xxxxxxxxxxxxxx xxxxxxxxxxxx xxxxxxxxx xxxxxx xxxxxxxxxxx xxx xxxxxx xxxxxxxxxxx xxxx xxx xxxxxxxxxxxxxxxxx xxxxxxxxxxx xxx xxxxxxxx xxxxxxxxxxxxx xxxxxxxxxxx xxxxxxxxxx xxxxxxxxxx xx xxxxxxxxx xxxxxxxxxxx xxxxxxx xxxxxxx xxxx xxxxxx xx xxxxxx xxxxxxxxxx xxx xxx xxxxxxxxxxx xxxxx xxxxxxxxxxxxxxxxx xxx xxxxxxxxxxxx xxx xxxxxxxxxx xxxxx xxxxxxxxxxxxxxxxx xxxxxxxxxxxxxxxxxxxxxxxxxxxxxxx xxx xxxxx xxxxxxxxxxxxxxxxxx xxxxxxxxxxxxxxxxxxxxxxxxx xxx xxxxxxxx xxxxxxxxxxxx xxxxxxxxxx xxxxxxxxxxxxxxxxxxxx

%%%%%%%%%%%%%%%%%%%%%%%%%%%%%%%%%%%%%%%%%%%%%%%%%%%%%%%%%%%%%%%%%%%%%%%%
% !TeX root = Hauptdatei.tex
% !TeX spellcheck = en_GB
%
%~~~~~~~~~~~~~~~~~~~~~~~~~~~~~~~~~~~~~~~~~~~~~~~~~~~~~~~~~~~~~~~~~~~~~~~~~~~~~~~~~~~~~~~~~~~~~~~~~~~~~~~
\chapter{Summary and Outlook}\label{chap:xxxx}
%~~~~~~~~~~~~~~~~~~~~~~~~~~~~~~~~~~~~~~~~~~~~~~~~~~~~~~~~~~~~~~~~~~~~~~~~~~~~~~~~~~~~~~~~~~~~~~~~~~~~~~~
\begin{otherlanguage}{english}
xxx xxxxxx xxxxxxxxxxx xxxxxxxxxx xxxx xxx xxxxxxxxxx xxxxxxxxxx xxx xxxxxxxxx xxxxxxx xxxxxx xxx xxxx x xxxx xxxxx xx xxxxxxx xxxx xxxxxx xx x xxxxxxxxxxxx xxxxxxxxxx xxxxxxxxx xxxx xxxxxxxxxx xxx xxxxxxxx xxxxxxxxx xx xxxxxxxxx xxxxx xx xxxxxxxxxxx xxxx xxxxx x xxxxxxxx xxxxxxxxxxxx xx xxxxxxxxxx xxx xxxxxxx xx x xxxxxx xxxxxxx xxxx xxxxxx xx xxx xxxxxxxxxxx xxxxxxxxxx xx xxxxxxxxxx xxxxxxxxxx xxxxxx xxxxxxx xxxxxxxxxxx xxxxxxxx xxxxxxxxxxxxxxx xxxxxxx xxxxxxxxxxxx xxxx xxxxxx xxxxxxxxxxxxx xxxxxxxxxxxx xx xxxxxxxxxxx xxxxxxxxxx xx xxxxx xx xxxxxxxx xxxxxxxxxxxx xxxxxxx xxxxxxxx xxxx xxxxxxx xxxxxxx xxx xxxxxx xx xxxx xxxxxxx xxx xxxxxxxxxx xxxxxxxx xxx xxxx xxxxxxxxx xx xxx xxxxxxxx xxxxxxxx xxxxxxx xx xxx xxxxxxxxxx xxxxxxxxxxxx xxxxxxx xx x xxxxxxxxxxx xxxxxxxxxxx

xxx xxxxxxxxxxx xx xxxxxxxxxx xxxxxxx xx xxxxxxxxxx xx xxx xxxxxxx xx xx xxxxxxxx xxxxxxxxxx xx x xxxxx xxxxx xxxxx xx x xxxxxx xx xxxxxx xxxxxxxxxxx xx xx xxxxxxx xx xxxxxxx xxxxxxxxx xxxxxxxxxxxx xxxxxxxx xx xxx xxxxxxxx xxxxxxx xxxxxx xxxxxxxxx xx xxx xxxx xx xxxxx xxxxxxxxxxxxxxxx xxxxx xxxxxxxxxxxx xxx xx xx xxxxxxxx xx xxxxxxxx xxx xxx xxxxxxxx xxx xxxx xxx xxx xxxxxxxxxxxx xxxx xxxxxx xx xxxxxxxxxx xxxx xxx xxxxxxxxxx xxx xxxxxxx xxxxxxxx xxxxxxxxxx xxx xxxxxxxxxxxxxxx xx xxxxx xxxxxxxxxxxx xxx xx xxxx xxxxxxxxxx xxxx xx xxx xxxxxxxx xxxxxxx xxxxxxxx xxxx xxxxx xxxxx xxxxxxx xx xxx xxxxxxxxx xx xxx xxxxxxxxxxx xxxxx xx xxxxxxx xxxxxxx xx xxxxx xxx xxxxxxxxx xx xxxxxxxx xx xxxxxxxx xxx xxxxxxx xxxxxxxxxxxxxxx xxxxxxx xxxxxxxxxx xx xx xxxxxxxxx xxxxx xxxxxxx xx xxxxxxx xxxxxxxxxxxxxxx xxxxxxxxx xx xxxxxxx xxxxxxxxxxx xxxxxx xxx xxxxxxxx xx xx xxxxxxxx xxxxxxxxx xx xxx xxxxxxx xx x xxxxxxxxx xxxxxxxxxxxx

xxx xxxxxxxxxxxxx xx xxx xxxxxxxxx xx xxxxx xxxxxxxx xxxxxxxxx xxxxx xxx xxxx xx xxxx xxxxx xxx xxx xx xx xxxxxxx x xxxxxxxxxxx xxx xxxxxxxxxx xxxxxxxxxxxx xxxxxxxx xxxxxxxx xxxxxxxxx xxxxx x xxxxxxxxxxxxxxxxxxxx xxxxxxxxxxx xxxxxxxxxx xxx xxxxxxx xxxx xxx xxxxxxx xx xxxxxxxxxx xxxxxxxx xx x xxxxxxx xxxx xxxxxx xxxx x xxxxxxxxxxxx xx xxxxxx xxx xxxxxxxxxxxx xxxxxx xx xxxxxxxxx xxx xxx xxxxxxx xxxxx xxxxxxxxxx xxxx xxxxxxx

xxx xxx xxxxxx xxxxxxxxxxxx xxxxxxxxxxx xxxx xxxxxxxxxx xxx xxxxxxxxx xx xxxxxxxxxx xxx xxxxxxx xxxxxxxxxxxxxxx xxxxxxx xxx xxxx xxxxx xxx xxxx xx xxxxxxxxx xxxx xxx xxx xxxxxxx xx xxxxxxxxxxx xxx xxxxxxxxxx xx xxxxxxxx xxxxxxxxx xxx xxx xxxxxxxx xxxxxxxxxxxxxxxxx xxxxxxxxxx xxxxxxxx xxxxxx xxxxxxxxxxx xxxxxxx xxxx xxxxxxxxxxx x xxxxxxxxxx xxxxxxxx xx xxxxxxxxx xxx xxxxxxxxxxxxxx xxxxxxxxx xxxxxxxxxxxxx xxxxxxxxx xxxxx xx xxxxxxxx xx xxxxxxxx xxxxxxxxxxx xxx xxxx xxxxx xx xxxx xxxx xx xxx xxxxxxxxxxx xx x xxxxxxxx xxxxxxxx xxx xxxxxxxxxxxxx xxx xxxxxxx xx x xxxxxxxxxxxxxxx xxxxxxxxxxxx xxxxxxxxx xxxxxxxxxxx xxxx xxx xxxxxxxxxxxxx xxxxxxxxxxxx xxxxxxxxxxx xx xxxxxxx xxxxxxx xxxxxxxxx xxxxxxxxxxx

\end{otherlanguage}
%%%%%%%%%%%%%%%%%%%%%%%%%%%%%%%%%%%%%%%%%%%%%%%%%%%%%%%%%%%%%%%%%%%%%%%%
% Anhang/Appendix
\appendix
% !TeX root = Hauptdatei.tex
% !TeX spellcheck = de_DE
%
\addsec{Ergänzungen zu Kapitel~\ref{chap:StandForschungTechnik}}\label{sec:AnhangXXX}
%~~~~~~~~~~~~~~~~~~~~~~~~~~~~~~~~~~~~~~~~~~~~~~~~~~~~~~~~~~~~~~~~~~~~~~~~~~~~~~~~~~~~~~~~~~~~~~~~~~~~~~~
xxxxxxxxxxxxx xxxxxxxxxxxxxx xxxxx xxx xxxxxxxxxxxxxx xxxxxxxxx xxxxxxxxxxxxxxx xxxxxxxxxxxx xxxxxxx xxx xxx xxxxxx xxxx xxx xxxxx xxxxxxx xxx xxxxxxxxxxx xxxxxxxxxxxxxxxxxx xxxxxxxxxxxxxxxxxxx xxxxxxxxxxxxxx xxxxxxxxxxxxxx xxx xxx xx xxxxxxx xxxxxxxxxxxxxxxxxx xxxxxxxxxxxx xxxxxxx xxxxxxxxxxxxxxxxxxx xxxxxxxxxxx xxxxx


% Output automated bibliography
\faupressprintbibliography

\backmatter
% !TeX root = Hauptdatei.tex
% !TeX spellcheck = en_GB
%
\chapter*{Abstract}
\markboth{\abstractname}{\abstractname}
\thispagestyle{empty}
Bei deutschsprachigen Arbeiten hier bitte einen englischsprachigen Abstract mit maximal circa 1500 Zeichen (einschließlich Leerzeichen) einfügen.

Bei englischsprachigen Arbeiten hier bitte eine deutschsprachige Kurzzusammenfassung mit maximal circa 1500 Zeichen (einschließlich Leerzeichen) einfügen.

\begin{otherlanguage}{english}%oder ngerman
xx xxxxxxxx xx x xxxxxxxxx xxxxxxxxxx xx xxxxxxxxxx xxx xxxxxxxxxx xxxxxxxxxxxxxx xxxxxxxxxxx xxxxxxxxxxxxxx xxxxxx xx xxxxxxxxxxxxx xxxxxxxxxxx xxxxxxxxxx xx xxxxxx xxxxxxxx xxxxxxxxxx xx xxx xxxxxxx xx x xxxxxx xxxxxxxxxxxxx xx x xxxxxxxx xxxxxx xx xxxxxxxxxxx xxxxxxxxx xxxxx xxxxxxx xxxxxxxx xxxxxxxxx xxx xxxxxxx xxxxxxxxx xxxxxxx xx xxxx xx xxxxxxxxxxx xxxxxxxxxx xxxx xxxxxxxx xx xxxxxxx xxxxxxx xxxxxxx xxxxxxxxx xxxxxxxxxxxx xx xxxxxxxx xxxx xxx xxxxxxx xx xxxxxxxxxx xxxxxxx xxxx xxx xxxxxxxxxxxxx xxxxxxxx xxx xx xxx xxxx xxxxxxxx xxx xxxxx xxxxx xx xxxxxxxxxxx xxxxxx xx xxx xxxxxxxx xxxxx xxxxxxxx xx xxx xxxxxxxxxxxxx

xx x xxxxxx xxxxxxxxxxx xx xxxx xxxxxxxxxx xxxx xxxxxx xxxxxxxxx x xxxxxxx xxxxxxxxx xxxxx xx xxxxxxxxx xxx xxxxxxx xxxxxxxxxxxxxxx xxxxxxxxxxxxxxx xxxx xxxxx xxxx xxxxxxxxxxxx xxx x xxxxxxxx xxxxxxxxxx xxxxxx xxxxxxxx xx xxx xxxxxxxxxx xxxx xxxxxxxxxxxxxx xxxx xxxxxxxxxx xx xxxxxxx xxxxxxxxxxxxx xxx xxxxxxxxxxxx xxxxxxx xxxxxxxxx xxxxxxxx xxxxxxxxxx xxxxxxxxx xxx xxxxxxxx xxxxxxx xx xxx xxxxxxxxx xxxxxxxxxx xx xxxxx xx x xxxx xxxxxxxxx xxxxxxxxxxx

xxxxx xx xxx xxxxxxx xx x xx xxxxxxxxx xxxxxxxx xxxx x xxxxxxxxxx xxxxxx xx xxxxxxxxxxx xxxxxxxxxxxxx xx xxxxxxxx xxx xxxxxxxxxx xxxxxxx xxxxxxxxxxxxxxxxxxxx xxxxxxxxx xxxxxxxx xxxxxxxxx xxxx xx xxxx xxxxxxxxx xx xxxxxx xxxxxxx xxxxxxxxxxxx xx xxxxxxxxx xxxxxxxxxxxxx xxx xxxxxxx xxxxxxxxxx xx xxx xxxxxx xxxxxxx xxxxx xx xxxxxxx xx xxxxxxxxxx xxxxxxxx xxxxxxxxxxxxx xxxxxxx xxxxxxxx xxxxxxxxx xxxxx xxxxxxxxx xxx xxxxxxxxx xxxxxxxxxxxxxx xx xxxxxxxxxxx xxxxxxxx xxx xxxxxxxx xx xxx xxxxx xxxxxxxx xxxxxxxxx xxxx xxxxx xxxxxxxxxxxxx xxxxxxx xxxxxxxxxxxx xxxx xx xxxxxxx xxxx xxxxxxxx xxxxxxxx xxxxxxxxxxx xxxx xxxxxxxx xxxx xx xxxxxxx xx xxxxx xx xxxxxx x xxxxxx xxxxxxxxx xxx xxxxxxx xxxxxxxx xxxxxxxxxx xxxxxxxxxxxxx xxxxxxxxxx xxxxxxxxxxx xxxx xxx xxxxx xx xxx xxxxxxxxxxxxxx xxxxxx xxxxxxx xxxxxxxxxxx xxxx xx xxxxxxxxx xxxxxxxxx xxx xxxxxxxxxxxxx xxxxxx xx xxx xxxxxx xxxxxxxxxxxx xx xxx xxxxxxxxxxxxxxx xx xxx xxxxxxxx xxxxxxxxxx xxxxxxxx xxxxxxxxxxx xxxxxxx xxx xxxxxxx xxx xx xxxxxx xxx xxxxxxxxxxxx xx xxx xxxxxxx xx xxxx xxxxxxx xxxxxxxxx xxxxxxxxxx xxxx xxxxxx xx xxx xxxxxxxx xxxxxxxxxxxxx x xxxxxxxxxx xxxx xx xxxx xxxx xxxx xxxx xx xxxxxxxxxxxx xxxx xxxx
\end{otherlanguage}
% !TeX root = Hauptdatei.tex
% !TeX spellcheck = de_DE
%
\chapter*{Kurzfassung}
\markboth{Kurzfassung}{Kurzfassung}
\thispagestyle{empty}
Klappentext (wird nicht im Innenteil des Manuskripts, sondern auf der Umschlagsrückseite abgedruckt)
Bei deutschsprachigen Arbeiten hier bitte eine deutschsprachige Kurzzusammenfassung mit maximal circa 1500 Zeichen (einschließlich Leerzeichen) für die Umschlagsrückseite mitliefern.
Bei englischsprachigen Arbeiten hier bitte einen englischsprachigen Abstract mit maximal circa 1500 Zeichen (einschließlich Leerzeichen) für die Umschlagsrückseite mitliefern:

xxxxxxx xxxxxxxxxxxxxx xxxxxx xxxxx xx xxx xxxxxxxx xx xxx xxxx xxxxxxxx xxxxxxxxxx xx xxxx-xxxxx xxxxxxxx xxxxxxxx. xxxxxx xxx xxxxxxx xxxxxxxxxxxxxx xx xxx xxxx xxxxx xxxx xxx xxxx-xxxx xxxxxx xxxxxxxx xxx xxx xxxxxx xxxxxxxx xxx xxxxxxxxxxxxx xx x xxxxxxx xxxxxxxxxxx xxx xxxxxxxx xxxx. xxxxx xx xxx xxxxxxxxxxxx xxxxxxxxxxxxx xx xxx xxxx xxxxxxxx xxxxxxx xx xxxxxxx xxx xxxx xxxxxx xx xxxxxxxxxx. xxxxxxx xxxxx xxxxx xxxxxx xx xxxxxxxxxxxx xx xxx xxxxxxxxxxxxxxx xx xxx xxxxxxxxx-xxxxx xxxxxxxxx xx xxx xxxx xxxxxxx. xxx xxx xx xxxx xxxxxxxxxxxxx xx xxx xxxxxxxxxx xx xxx xxxxxxx xxxxxx. xxxxxxxxxxxx, xxx xxxxxxxxxxx xx x xxxxx xxxxxxxx xx xxx xxxxxxxxxxxx xxxxxxxxxx xxxxx xxxxxxxxx xxxxxxxxxx xxxxxx xx xxxxxxxxxxxx.

Der gelieferte Text wird vom Verlag als Klappentext auf der hinteren Umschlagseite platziert, über der ISBN und dem Barcode.

\end{document}
